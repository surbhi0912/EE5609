\documentclass[journal,12pt]{IEEEtran}

\usepackage{setspace}
\usepackage{gensymb}


\singlespacing

\usepackage[cmex10]{amsmath}
%\usepackage{amsthm}
%\interdisplaylinepenalty=2500
%\savesymbol{iint}
%\usepackage{txfonts}
%\restoresymbol{TXF}{iint}
%\usepackage{wasysym}
\usepackage{amsthm}

\usepackage{mathrsfs}
\usepackage{txfonts}
\usepackage{stfloats}
\usepackage{bm}
\usepackage{cite}
\usepackage{cases}
\usepackage{subfig}

\usepackage{longtable}
\usepackage{multirow}

\usepackage{enumitem}
\usepackage{mathtools}
\usepackage{steinmetz}
\usepackage{tikz}
\usepackage{circuitikz}
\usepackage{verbatim}
\usepackage{tfrupee}
\usepackage[breaklinks=true]{hyperref}

\usepackage{tkz-euclide} %loads TikZ and tkz-base

\usetikzlibrary{calc,math}
\usepackage{listings}
    \usepackage{color}                                          
    \usepackage{array}                                          
    \usepackage{longtable}                                      
    \usepackage{calc}                                           
    \usepackage{multirow}                                       
    \usepackage{hhline}                                         
    \usepackage{ifthen}
    \usepackage{lscape}     
\usepackage{multicol}
\usepackage{chngcntr}

\DeclareMathOperator*{\Res}{Res}

\renewcommand\thesection{\arabic{section}}
\renewcommand\thesubsection{\thesection.\arabic{subsection}}
\renewcommand\thesubsubsection{\thesubsection.\arabic{subsubsection}}

\renewcommand\thesectiondis{\arabic{section}}
\renewcommand\thesubsectiondis{\thesectiondis.\arabic{subsection}}
\renewcommand\thesubsubsectiondis{\thesubsectiondis.\arabic{subsubsection}}

\hyphenation{op-tical net-works semi-conduc-tor}
\def\inputGnumericTable{}                                 %%

\lstset{
%language=C,
frame=single, 
breaklines=true,
columns=fullflexible
}

\begin{document}

\newtheorem{theorem}{Theorem}[section]
\newtheorem{problem}{Problem}
\newtheorem{proposition}{Proposition}[section]
\newtheorem{lemma}{Lemma}[section]
\newtheorem{corollary}[theorem]{Corollary}
\newtheorem{example}{Example}[section]
\newtheorem{definition}[problem]{Definition}

\newcommand{\BEQA}{\begin{eqnarray}}
\newcommand{\EEQA}{\end{eqnarray}}
\newcommand{\define}{\stackrel{\triangle}{=}}

\bibliographystyle{IEEEtran}

\providecommand{\mbf}{\mathbf}
\providecommand{\pr}[1]{\ensuremath{\Pr\left(#1\right)}}
\providecommand{\qfunc}[1]{\ensuremath{Q\left(#1\right)}}
\providecommand{\sbrak}[1]{\ensuremath{{}\left[#1\right]}}
\providecommand{\lsbrak}[1]{\ensuremath{{}\left[#1\right.}}
\providecommand{\rsbrak}[1]{\ensuremath{{}\left.#1\right]}}
\providecommand{\brak}[1]{\ensuremath{\left(#1\right)}}
\providecommand{\lbrak}[1]{\ensuremath{\left(#1\right.}}
\providecommand{\rbrak}[1]{\ensuremath{\left.#1\right)}}
\providecommand{\cbrak}[1]{\ensuremath{\left\{#1\right\}}}
\providecommand{\lcbrak}[1]{\ensuremath{\left\{#1\right.}}
\providecommand{\rcbrak}[1]{\ensuremath{\left.#1\right\}}}
\theoremstyle{remark}
\newtheorem{rem}{Remark}
\newcommand{\sgn}{\mathop{\mathrm{sgn}}}
\providecommand{\abs}[1]{\left\vert#1\right\vert}
\providecommand{\res}[1]{\Res\displaylimits_{#1}} 
\providecommand{\norm}[1]{\left\lVert#1\right\rVert}
%\providecommand{\norm}[1]{\lVert#1\rVert}
\providecommand{\mtx}[1]{\mathbf{#1}}
\providecommand{\mean}[1]{E\left[ #1 \right]}
\providecommand{\fourier}{\overset{\mathcal{F}}{ \rightleftharpoons}}
%\providecommand{\hilbert}{\overset{\mathcal{H}}{ \rightleftharpoons}}
\providecommand{\system}{\overset{\mathcal{H}}{ \longleftrightarrow}}
	%\newcommand{\solution}[2]{\textbf{Solution:}{#1}}
\newcommand{\solution}{\noindent \textbf{Solution: }}
\newcommand{\cosec}{\,\text{cosec}\,}
\providecommand{\dec}[2]{\ensuremath{\overset{#1}{\underset{#2}{\gtrless}}}}
\newcommand{\myvec}[1]{\ensuremath{\begin{pmatrix}#1\end{pmatrix}}}
\newcommand{\mydet}[1]{\ensuremath{\begin{vmatrix}#1\end{vmatrix}}}

\numberwithin{equation}{subsection}

\makeatletter
\@addtoreset{figure}{problem}
\makeatother

\let\StandardTheFigure\thefigure
\let\vec\mathbf

\renewcommand{\thefigure}{\theproblem}

\def\putbox#1#2#3{\makebox[0in][l]{\makebox[#1][l]{}\raisebox{\baselineskip}[0in][0in]{\raisebox{#2}[0in][0in]{#3}}}}
     \def\rightbox#1{\makebox[0in][r]{#1}}
     \def\centbox#1{\makebox[0in]{#1}}
     \def\topbox#1{\raisebox{-\baselineskip}[0in][0in]{#1}}
     \def\midbox#1{\raisebox{-0.5\baselineskip}[0in][0in]{#1}}
\vspace{3cm}
\begin{center}
\huge Assignment 14\\
\large Surbhi Agarwal\\
\end{center}
\renewcommand{\thefigure}{\theenumi}
\renewcommand{\thetable}{\theenumi}

\begin{abstract}
This document shows some properties of matrices, their inverse and determinant.
\end{abstract}

\section{Problem}
The matrix $\vec{A} = \myvec{5 & 9 & 8\\ 1 & 8 & 2\\ 9 & 1 & 0}$ satisfies:
\begin{enumerate}
    \item $\vec{A}$ is invertible and the inverse has all integer entries.
    \item det$(\vec{A})$ is odd.
    \item det$(\vec{A})$ is divisible by 13
    \item det$(\vec{A})$ has atleast two prime divisors.
\end{enumerate}
\section{Solution}
Performing some elementary row operations on the given matrix,
\begin{align}
    \myvec{5 & 9 & 8\\ 1 & 8 & 2\\ 9 & 1 & 0} \xleftrightarrow[R_3 \leftarrow R_3 - \frac{1}{9}R_1]{R_2 \leftarrow R_2 - \frac{1}{5}R_1} \myvec{5 & 9 & 8\\ 0 & \frac{31}{5} & \frac{2}{5} \\ 0 & \frac{-76}{5} & \frac{-72}{5}}\\
    \xleftrightarrow{R_3 \leftarrow R_3 + \frac{76}{31}R_2} \myvec{5 & 9 & 8\\ 0 & \frac{31}{5} & \frac{2}{5} \\ 0 & 0 & \frac{-416}{31}}
\end{align}
After obtaining a triangular form of the matrix, we can say
\begin{align}
    \mydet{\vec{A}} & = \text{product of diagonal entries of the triangular 
    matrix}\\&
    = 5 \times \frac{31}{5} \times \frac{-416}{31} = -416 \label{eq:finaldet}
\end{align}
\renewcommand{\thetable}{1}
\begin{longtable}{|l|l|}
    \hline
        & \\
        1. $\vec{A}$ is invertible and the inverse has all integer entries & $det(\vec{A}) \neq 0$, hence $\vec{A}$ is invertible.\\
        & \\
        & $\vec{A}$ is an integer matrix and has all integer entries.\\
        & $\therefore det(\vec{A})$ is an integer\\
        & \\
        & If $\vec{A}^{-1}$ is an integer matrix and has all integer entries,\\ & then $det(\vec{A}^{-1})$ will be an integer.\\
        & \\
        & $\vec{A}\vec{A}^{-1} = \vec{I}_{3 \times 3}$\\
        & $det(\vec{A}\vec{A}^{-1}) = det(\vec{I}) = 1$\\
        & $det(\vec{A})det(\vec{A}^{-1}) = 1$\\
        & This is possible only when $det(\vec{A}) = \pm 1$\\
        & \\
    \hline
        & \\
        & But we have seen that $det(\vec{A}) = -416$, hence\\
        & $\vec{A}^{-1}$ does not have all integer entries.\\
        & Given option is false.\\
        & \\
    \hline
        & \\
        2. det$(\vec{A})$ is odd & False, as seen from \eqref{eq:finaldet}\\
        & \\
    \hline
        & \\
        3. det$(\vec{A})$ is divisible by 13 & True. Since $det(\vec{A}) = -416$,\\
        & which is divisible by 13.\\
        & \\
    \hline
        & \\
        4. det$(\vec{A})$ has atleast two prime divisors & True. As seen from \eqref{eq:finaldet},\\
        & $det(\vec{A}) = -416 = -1 \times 2^5 \times 13$\\
        & so $det(\vec{A})$ has atleast 2 prime divisors.\\
        & \\
    \hline
    \caption{Verifying with given options}
    \label{tab:verify}
\end{longtable}
\end{document}