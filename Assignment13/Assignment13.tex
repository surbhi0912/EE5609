\documentclass[journal,12pt]{IEEEtran}

\usepackage{setspace}
\usepackage{gensymb}


\singlespacing

\usepackage[cmex10]{amsmath}
%\usepackage{amsthm}
%\interdisplaylinepenalty=2500
%\savesymbol{iint}
%\usepackage{txfonts}
%\restoresymbol{TXF}{iint}
%\usepackage{wasysym}
\usepackage{amsthm}

\usepackage{mathrsfs}
\usepackage{txfonts}
\usepackage{stfloats}
\usepackage{bm}
\usepackage{cite}
\usepackage{cases}
\usepackage{subfig}

\usepackage{longtable}
\usepackage{multirow}

\usepackage{enumitem}
\usepackage{mathtools}
\usepackage{steinmetz}
\usepackage{tikz}
\usepackage{circuitikz}
\usepackage{verbatim}
\usepackage{tfrupee}
\usepackage[breaklinks=true]{hyperref}

\usepackage{tkz-euclide} %loads TikZ and tkz-base

\usetikzlibrary{calc,math}
\usepackage{listings}
    \usepackage{color}                                          
    \usepackage{array}                                          
    \usepackage{longtable}                                      
    \usepackage{calc}                                           
    \usepackage{multirow}                                       
    \usepackage{hhline}                                         
    \usepackage{ifthen}
    \usepackage{lscape}     
\usepackage{multicol}
\usepackage{chngcntr}

\DeclareMathOperator*{\Res}{Res}

\renewcommand\thesection{\arabic{section}}
\renewcommand\thesubsection{\thesection.\arabic{subsection}}
\renewcommand\thesubsubsection{\thesubsection.\arabic{subsubsection}}

\renewcommand\thesectiondis{\arabic{section}}
\renewcommand\thesubsectiondis{\thesectiondis.\arabic{subsection}}
\renewcommand\thesubsubsectiondis{\thesubsectiondis.\arabic{subsubsection}}

\hyphenation{op-tical net-works semi-conduc-tor}
\def\inputGnumericTable{}                                 %%

\lstset{
%language=C,
frame=single, 
breaklines=true,
columns=fullflexible
}

\begin{document}

\newtheorem{theorem}{Theorem}[section]
\newtheorem{problem}{Problem}
\newtheorem{proposition}{Proposition}[section]
\newtheorem{lemma}{Lemma}[section]
\newtheorem{corollary}[theorem]{Corollary}
\newtheorem{example}{Example}[section]
\newtheorem{definition}[problem]{Definition}

\newcommand{\BEQA}{\begin{eqnarray}}
\newcommand{\EEQA}{\end{eqnarray}}
\newcommand{\define}{\stackrel{\triangle}{=}}

\bibliographystyle{IEEEtran}

\providecommand{\mbf}{\mathbf}
\providecommand{\pr}[1]{\ensuremath{\Pr\left(#1\right)}}
\providecommand{\qfunc}[1]{\ensuremath{Q\left(#1\right)}}
\providecommand{\sbrak}[1]{\ensuremath{{}\left[#1\right]}}
\providecommand{\lsbrak}[1]{\ensuremath{{}\left[#1\right.}}
\providecommand{\rsbrak}[1]{\ensuremath{{}\left.#1\right]}}
\providecommand{\brak}[1]{\ensuremath{\left(#1\right)}}
\providecommand{\lbrak}[1]{\ensuremath{\left(#1\right.}}
\providecommand{\rbrak}[1]{\ensuremath{\left.#1\right)}}
\providecommand{\cbrak}[1]{\ensuremath{\left\{#1\right\}}}
\providecommand{\lcbrak}[1]{\ensuremath{\left\{#1\right.}}
\providecommand{\rcbrak}[1]{\ensuremath{\left.#1\right\}}}
\theoremstyle{remark}
\newtheorem{rem}{Remark}
\newcommand{\sgn}{\mathop{\mathrm{sgn}}}
\providecommand{\abs}[1]{\left\vert#1\right\vert}
\providecommand{\res}[1]{\Res\displaylimits_{#1}} 
\providecommand{\norm}[1]{\left\lVert#1\right\rVert}
%\providecommand{\norm}[1]{\lVert#1\rVert}
\providecommand{\mtx}[1]{\mathbf{#1}}
\providecommand{\mean}[1]{E\left[ #1 \right]}
\providecommand{\fourier}{\overset{\mathcal{F}}{ \rightleftharpoons}}
%\providecommand{\hilbert}{\overset{\mathcal{H}}{ \rightleftharpoons}}
\providecommand{\system}{\overset{\mathcal{H}}{ \longleftrightarrow}}
	%\newcommand{\solution}[2]{\textbf{Solution:}{#1}}
\newcommand{\solution}{\noindent \textbf{Solution: }}
\newcommand{\cosec}{\,\text{cosec}\,}
\providecommand{\dec}[2]{\ensuremath{\overset{#1}{\underset{#2}{\gtrless}}}}
\newcommand{\myvec}[1]{\ensuremath{\begin{pmatrix}#1\end{pmatrix}}}
\newcommand{\mydet}[1]{\ensuremath{\begin{vmatrix}#1\end{vmatrix}}}

\numberwithin{equation}{subsection}

\makeatletter
\@addtoreset{figure}{problem}
\makeatother

\let\StandardTheFigure\thefigure
\let\vec\mathbf

\renewcommand{\thefigure}{\theproblem}

\def\putbox#1#2#3{\makebox[0in][l]{\makebox[#1][l]{}\raisebox{\baselineskip}[0in][0in]{\raisebox{#2}[0in][0in]{#3}}}}
     \def\rightbox#1{\makebox[0in][r]{#1}}
     \def\centbox#1{\makebox[0in]{#1}}
     \def\topbox#1{\raisebox{-\baselineskip}[0in][0in]{#1}}
     \def\midbox#1{\raisebox{-0.5\baselineskip}[0in][0in]{#1}}
\vspace{3cm}
\begin{center}
\huge Assignment 13\\
\large Surbhi Agarwal\\
\end{center}
\renewcommand{\thefigure}{\theenumi}
\renewcommand{\thetable}{\theenumi}

\begin{abstract}
This document illustrates properties of subspaces of a vectorspace.
\end{abstract}

\section{Problem}
For arbitrary subspaces, $U$, $V$ and $W$ of a finite dimensional vectorspace, which of the following hold :
\begin{enumerate}
    \item $U \cap (V+W) \subset (U \cap V) + (U \cap W)$
    \item $U \cap (V+W) \supset (U \cap V) + (U \cap W)$
    \item $(U \cap V) + W \subset (U+W) \cap (V+W)$
    \item $(U \cap V) + W \supset (U+W) \cap (V+W)$
\end{enumerate}
\section{Solution}
\renewcommand{\thetable}{1}
\begin{longtable}{|l|l|}
    \hline
        & \\
        1. $U \cap (V+W) \subset (U \cap V) + (U \cap W)$ & False.\\
        & \\
        & Counter Example:\\
        & Let $\vec{u}_1 = (\vec{v}_1 + \vec{w}_1) \in U \cap (V+W)$ such that\\
        & $(\vec{v}_1 + \vec{w}_1) \in U, \vec{v}_1 \in V, \vec{w}_1 \in W$\\
        & \\
        & But since $\vec{w}_1 \not\in V$, hence $\vec{v}_1 + \vec{w}_1 \not\in V$\\
        & $\implies (\vec{v}_1 + \vec{w}_1) \not\in (U\cap V)$\\
        & Also, $\vec{v}_1, \vec{w}_1 \not\in U \implies \vec{v}_1, \vec{w}_1 \not\in (U \cap V)$\\
        &\\
        & And since $\vec{v}_1 \not\in W$, hence $\vec{v}_1 + \vec{w}_1 \not\in W$\\
        & $\implies (\vec{v}_1 + \vec{w}_1) \not\in (U\cap W)$\\
        & Also, $\vec{v}_1, \vec{w}_1 \not\in U \implies \vec{v}_1, \vec{w}_1 \not\in (U \cap W)$\\
        & \\
        & Therefore, $(\vec{v}_1 + \vec{w}_1) \not\in (U\cap V) + (U \cap W)$\\
        & \\
        & There exists an element in LHS that does not belong to RHS.\\
        & $\therefore U \cap (V+W) \not\subset (U \cap V) + (U \cap W)$\\
        & \\
    \hline
        & \\
        2. $U \cap (V+W) \supset (U \cap V) + (U \cap W)$ & Let $(\vec{u}_1 + \vec{u}_2) \in (U \cap V) + (U \cap W)$\\
        & such that $\vec{u}_1 \in U \cap V$\\
        & and $\vec{u}_2 \in U \cap W$\\
        & $\implies \vec{u}_1 \in U, V$ and $\vec{u}_2 \in U,W$\\
        & \\
        & Since $\vec{u}_1 \in V, \vec{u}_2 \in W$\\
        & $\implies (\vec{u}_1 + \vec{u}_2) \in (V+W)$\\
        & And since $\vec{u}_1, \vec{u}_2 \in U$\\
        & $\implies (\vec{u}_1 + \vec{u}_2) \in U$\\
        & $\therefore (\vec{u}_1 + \vec{u}_2) \in U \cap (V+W)$\\
        & \\
    \hline
        & \\
        & So, $(\vec{u}_1 + \vec{u}_2) \in (U \cap V) + (U \cap W) \implies (\vec{u}_1 + \vec{u}_2) \in U \cap (V+W)$\\
        & Hence, $U \cap (V+W) \supset (U \cap V) + (U \cap W)$\\
        & \\
        & The given option is true.\\
        & \\
    \hline
        & \\
        3. $(U \cap V) + W \subset (U+W) \cap (V+W)$ & Let $(\vec{u}_1 + \vec{w}_1) \in (U \cap V) + W$, such that\\
        & $\vec{u}_1 \in (U \cap V)$ and $\vec{w}_1 \in W$\\
        & Since, $\vec{u}_1 \in (U \cap V), \implies \vec{u}_1 \in U, V$\\
        & \\
        & Now, since $\vec{u}_1 \in U, \vec{w}_1 \in W$\\
        & $(\vec{u}_1 + \vec{w}_1) \in (U+W)$\\
        & And since, $\vec{u}_1 \in V, \vec{w}_1 \in W$\\
        & $(\vec{u}_1 + \vec{w}_1) \in (V+W)$\\
        & $\therefore (\vec{u}_1 + \vec{w}_1) \in (U+W) \cap (V+W)$\\
        & \\
        & Hence, $(\vec{u}_1 + \vec{w}_1) \in (U \cap V) + W \implies (\vec{u}_1 + \vec{w}_1) \in (U+W) \cap (V+W)$\\
        & $(U \cap V) + W \subset (U+W) \cap (V+W)$\\
        & \\
        & The given option is true.\\
        & \\
    \hline
        & \\
        4. $(U \cap V) + W \supset (U+W) \cap (V+W)$ & False.\\
        & \\
        & Counter Example:\\
        & Let $\vec{u}_1 = \vec{v}_1 + \vec{w}_1 \in U$\\
        & $\vec{v}_1 \in V, \vec{w}_1 \in W$\\
        & \\
        & Then, since $\vec{v}_1 + \vec{w}_1 \in U \implies \vec{v}_1 + \vec{w}_1 \in U+W$\\
        & And since, $\vec{v}_1 \in V, \vec{w}_1 \in W \implies \vec{v}_1 + \vec{w}_1 \in V+W$\\
        & $\therefore \vec{v}_1 + \vec{w}_1 \in (U+W) \cap (V+W)$\\
        & \\
        & Now, since $\vec{w}_1 \not\in V \implies \vec{v}_1 + \vec{w}_1 \not\in V$\\
        & $\implies \vec{v}_1 + \vec{w}_1 \not\in U \cap V$\\
        & Also, $\vec{v}_1 \not\in U \implies \vec{v}_1 \not\in (U \cap V)$\\
        & \\
        & And since, $\vec{v}_1 \not\in W \implies \vec{v}_1 + \vec{w}_1 \not\in W$\\
        & \\
        & $\implies \vec{v}_1 + \vec{w}_1 \not\in (U \cap V) + W$\\
        & \\
        & There exists an element in RHS that does not exist in LHS\\
        & $\therefore (U \cap V) + W \not\supset (U+W) \cap (V+W)$\\
        & \\
    \hline
    \caption{Proving properties of subspaces of a vectorspace}
    \label{tab:proof}
\end{longtable}
\end{document}