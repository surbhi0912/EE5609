\documentclass[journal,12pt,twocolumn]{IEEEtran}

\usepackage{setspace}
\usepackage{gensymb}


\singlespacing

\usepackage[cmex10]{amsmath}
%\usepackage{amsthm}
%\interdisplaylinepenalty=2500
%\savesymbol{iint}
%\usepackage{txfonts}
%\restoresymbol{TXF}{iint}
%\usepackage{wasysym}
\usepackage{amsthm}

\usepackage{mathrsfs}
\usepackage{txfonts}
\usepackage{stfloats}
\usepackage{bm}
\usepackage{cite}
\usepackage{cases}
\usepackage{subfig}

\usepackage{longtable}
\usepackage{multirow}

\usepackage{enumitem}
\usepackage{mathtools}
\usepackage{steinmetz}
\usepackage{tikz}
\usepackage{circuitikz}
\usepackage{verbatim}
\usepackage{tfrupee}
\usepackage[breaklinks=true]{hyperref}

\usepackage{tkz-euclide} %loads TikZ and tkz-base

\usetikzlibrary{calc,math}
\usepackage{listings}
    \usepackage{color}                                          
    \usepackage{array}                                          
    \usepackage{longtable}                                      
    \usepackage{calc}                                           
    \usepackage{multirow}                                       
    \usepackage{hhline}                                         
    \usepackage{ifthen}
    \usepackage{lscape}     
\usepackage{multicol}
\usepackage{chngcntr}

\DeclareMathOperator*{\Res}{Res}

\renewcommand\thesection{\arabic{section}}
\renewcommand\thesubsection{\thesection.\arabic{subsection}}
\renewcommand\thesubsubsection{\thesubsection.\arabic{subsubsection}}

\renewcommand\thesectiondis{\arabic{section}}
\renewcommand\thesubsectiondis{\thesectiondis.\arabic{subsection}}
\renewcommand\thesubsubsectiondis{\thesubsectiondis.\arabic{subsubsection}}

\hyphenation{op-tical net-works semi-conduc-tor}
\def\inputGnumericTable{}                                 %%

\lstset{
%language=C,
frame=single, 
breaklines=true,
columns=fullflexible
}

\begin{document}

\newtheorem{theorem}{Theorem}[section]
\newtheorem{problem}{Problem}
\newtheorem{proposition}{Proposition}[section]
\newtheorem{lemma}{Lemma}[section]
\newtheorem{corollary}[theorem]{Corollary}
\newtheorem{example}{Example}[section]
\newtheorem{definition}[problem]{Definition}

\newcommand{\BEQA}{\begin{eqnarray}}
\newcommand{\EEQA}{\end{eqnarray}}
\newcommand{\define}{\stackrel{\triangle}{=}}

\bibliographystyle{IEEEtran}

\providecommand{\mbf}{\mathbf}
\providecommand{\pr}[1]{\ensuremath{\Pr\left(#1\right)}}
\providecommand{\qfunc}[1]{\ensuremath{Q\left(#1\right)}}
\providecommand{\sbrak}[1]{\ensuremath{{}\left[#1\right]}}
\providecommand{\lsbrak}[1]{\ensuremath{{}\left[#1\right.}}
\providecommand{\rsbrak}[1]{\ensuremath{{}\left.#1\right]}}
\providecommand{\brak}[1]{\ensuremath{\left(#1\right)}}
\providecommand{\lbrak}[1]{\ensuremath{\left(#1\right.}}
\providecommand{\rbrak}[1]{\ensuremath{\left.#1\right)}}
\providecommand{\cbrak}[1]{\ensuremath{\left\{#1\right\}}}
\providecommand{\lcbrak}[1]{\ensuremath{\left\{#1\right.}}
\providecommand{\rcbrak}[1]{\ensuremath{\left.#1\right\}}}
\theoremstyle{remark}
\newtheorem{rem}{Remark}
\newcommand{\sgn}{\mathop{\mathrm{sgn}}}
\providecommand{\abs}[1]{\left\vert#1\right\vert}
\providecommand{\res}[1]{\Res\displaylimits_{#1}} 
\providecommand{\norm}[1]{\left\lVert#1\right\rVert}
%\providecommand{\norm}[1]{\lVert#1\rVert}
\providecommand{\mtx}[1]{\mathbf{#1}}
\providecommand{\mean}[1]{E\left[ #1 \right]}
\providecommand{\fourier}{\overset{\mathcal{F}}{ \rightleftharpoons}}
%\providecommand{\hilbert}{\overset{\mathcal{H}}{ \rightleftharpoons}}
\providecommand{\system}{\overset{\mathcal{H}}{ \longleftrightarrow}}
	%\newcommand{\solution}[2]{\textbf{Solution:}{#1}}
\newcommand{\solution}{\noindent \textbf{Solution: }}
\newcommand{\cosec}{\,\text{cosec}\,}
\providecommand{\dec}[2]{\ensuremath{\overset{#1}{\underset{#2}{\gtrless}}}}
\newcommand{\myvec}[1]{\ensuremath{\begin{pmatrix}#1\end{pmatrix}}}
\newcommand{\mydet}[1]{\ensuremath{\begin{vmatrix}#1\end{vmatrix}}}

\numberwithin{equation}{subsection}

\makeatletter
\@addtoreset{figure}{problem}
\makeatother

\let\StandardTheFigure\thefigure
\let\vec\mathbf

\renewcommand{\thefigure}{\theproblem}

\def\putbox#1#2#3{\makebox[0in][l]{\makebox[#1][l]{}\raisebox{\baselineskip}[0in][0in]{\raisebox{#2}[0in][0in]{#3}}}}
     \def\rightbox#1{\makebox[0in][r]{#1}}
     \def\centbox#1{\makebox[0in]{#1}}
     \def\topbox#1{\raisebox{-\baselineskip}[0in][0in]{#1}}
     \def\midbox#1{\raisebox{-0.5\baselineskip}[0in][0in]{#1}}
\vspace{3cm}

\title{Assignment 10}
\author{Surbhi Agarwal}

\maketitle

\newpage

%\tableofcontents

\bigskip

\renewcommand{\thefigure}{\theenumi}
\renewcommand{\thetable}{\theenumi}

Download all latex-tikz codes from 
%
\begin{lstlisting}
https://github.com/surbhi0912/EE5609/
\end{lstlisting}
%
\section{Problem}
Given that there are real constants $a, b, c, d$ such that the identity
\begin{align}\label{nonumber}
    \lambda x^2 + 2xy + y^2 = (ax+by)^2 + (cx+dy)^2
\end{align}
holds for all $x, y \in \mathbb R$. This implies
\begin{enumerate}
    \item $\lambda = -5$
    \item $\lambda \geq 1$
    \item $0 < \lambda < 1$
    \item there is no such $\lambda \in \mathbb R$
\end{enumerate}
\section{Solution}
Given that
\begin{align}
    \lambda x^2+2xy+y^2 = (ax+by)^2+(cx+dy)^2
    %\lambda x^2 + 2xy + y^2 =  (a^2+c^2)x^2 + 2(ab + cd)xy + (b^2+d^2)y^2
    \intertext{Arranging this in form of a matrix,}
    \myvec{x & y}\myvec{\lambda & 1 \\ 1 & 1}\myvec{x \\ y} = \myvec{x & y}\myvec{a^2 + c^2 & ab+cd \\ ab+cd & b^2+d^2}\myvec{x \\ y}
\end{align}
From this, we get
\begin{align}
    \lambda = a^2+c^2 \label{eq:oflambda}\\
    ab+cd = 1\label{eq:abandcd}\\
    b^2 + d^2 = 1
    \intertext{Let}
    b = \cos \theta, d = \sin \theta \label{eq:bandd}\\
    \therefore \cos^2 \theta + \sin^2 \theta = 1, \quad \forall \theta \in \mathbb R
    \intertext{Substituting \eqref{eq:bandd} in \eqref{eq:abandcd}}
    a\cos\theta + c\sin\theta = 1 \label{eq:newabandcd}\\
    \text{let} a = k \cos \beta, c = k \sin \beta \label{eq:aandc}\\
    \intertext{Substituting this in $\eqref{eq:newabandcd}$}
    k\cos\beta\sin\theta + k\sin\beta\cos\theta = 1\\
    \implies k \sin (\theta + \beta) = 1 \label{eq:k}
\end{align}
Substituting \eqref{eq:aandc} in \eqref{eq:oflambda},
\begin{align}
    \lambda & = k^2 \cos^2\beta + k^2 \sin^2 \beta \\ &
    = k^2 (\cos^2\beta + \sin^2 \beta)\\ &
    = k^2 \label{eq:lambdaandk}
\end{align}
Now, since $\abs{\sin(\theta+\beta)} \leq 1$, then from \eqref{eq:k}, we get $\abs{k} \geq 1$, hence $k^2 \geq 1$. Using this in \eqref{eq:lambdaandk},
\begin{align}
    \lambda \geq 1
\end{align}
So from the given options, option 2) $\lambda \geq 1$ is correct.
\end{document}