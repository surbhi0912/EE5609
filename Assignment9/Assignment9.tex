\documentclass[journal,12pt]{IEEEtran}

\usepackage{setspace}
\usepackage{gensymb}


\singlespacing

\usepackage[cmex10]{amsmath}
%\usepackage{amsthm}
%\interdisplaylinepenalty=2500
%\savesymbol{iint}
%\usepackage{txfonts}
%\restoresymbol{TXF}{iint}
%\usepackage{wasysym}
\usepackage{amsthm}

\usepackage{mathrsfs}
\usepackage{txfonts}
\usepackage{stfloats}
\usepackage{bm}
\usepackage{cite}
\usepackage{cases}
\usepackage{subfig}

\usepackage{longtable}
\usepackage{multirow}

\usepackage{enumitem}
\usepackage{mathtools}
\usepackage{steinmetz}
\usepackage{tikz}
\usepackage{circuitikz}
\usepackage{verbatim}
\usepackage{tfrupee}
\usepackage[breaklinks=true]{hyperref}

\usepackage{tkz-euclide} %loads TikZ and tkz-base

\usetikzlibrary{calc,math}
\usepackage{listings}
    \usepackage{color}                                          
    \usepackage{array}                                          
    \usepackage{longtable}                                      
    \usepackage{calc}                                           
    \usepackage{multirow}                                       
    \usepackage{hhline}                                         
    \usepackage{ifthen}
    \usepackage{lscape}     
\usepackage{multicol}
\usepackage{chngcntr}

\DeclareMathOperator*{\Res}{Res}

\renewcommand\thesection{\arabic{section}}
\renewcommand\thesubsection{\thesection.\arabic{subsection}}
\renewcommand\thesubsubsection{\thesubsection.\arabic{subsubsection}}

\renewcommand\thesectiondis{\arabic{section}}
\renewcommand\thesubsectiondis{\thesectiondis.\arabic{subsection}}
\renewcommand\thesubsubsectiondis{\thesubsectiondis.\arabic{subsubsection}}

\hyphenation{op-tical net-works semi-conduc-tor}
\def\inputGnumericTable{}                                 %%

\lstset{
%language=C,
frame=single, 
breaklines=true,
columns=fullflexible
}

\begin{document}

\newtheorem{theorem}{Theorem}[section]
\newtheorem{problem}{Problem}
\newtheorem{proposition}{Proposition}[section]
\newtheorem{lemma}{Lemma}[section]
\newtheorem{corollary}[theorem]{Corollary}
\newtheorem{example}{Example}[section]
\newtheorem{definition}[problem]{Definition}

\newcommand{\BEQA}{\begin{eqnarray}}
\newcommand{\EEQA}{\end{eqnarray}}
\newcommand{\define}{\stackrel{\triangle}{=}}

\bibliographystyle{IEEEtran}

\providecommand{\mbf}{\mathbf}
\providecommand{\pr}[1]{\ensuremath{\Pr\left(#1\right)}}
\providecommand{\qfunc}[1]{\ensuremath{Q\left(#1\right)}}
\providecommand{\sbrak}[1]{\ensuremath{{}\left[#1\right]}}
\providecommand{\lsbrak}[1]{\ensuremath{{}\left[#1\right.}}
\providecommand{\rsbrak}[1]{\ensuremath{{}\left.#1\right]}}
\providecommand{\brak}[1]{\ensuremath{\left(#1\right)}}
\providecommand{\lbrak}[1]{\ensuremath{\left(#1\right.}}
\providecommand{\rbrak}[1]{\ensuremath{\left.#1\right)}}
\providecommand{\cbrak}[1]{\ensuremath{\left\{#1\right\}}}
\providecommand{\lcbrak}[1]{\ensuremath{\left\{#1\right.}}
\providecommand{\rcbrak}[1]{\ensuremath{\left.#1\right\}}}
\theoremstyle{remark}
\newtheorem{rem}{Remark}
\newcommand{\sgn}{\mathop{\mathrm{sgn}}}
\providecommand{\abs}[1]{\left\vert#1\right\vert}
\providecommand{\res}[1]{\Res\displaylimits_{#1}} 
\providecommand{\norm}[1]{\left\lVert#1\right\rVert}
%\providecommand{\norm}[1]{\lVert#1\rVert}
\providecommand{\mtx}[1]{\mathbf{#1}}
\providecommand{\mean}[1]{E\left[ #1 \right]}
\providecommand{\fourier}{\overset{\mathcal{F}}{ \rightleftharpoons}}
%\providecommand{\hilbert}{\overset{\mathcal{H}}{ \rightleftharpoons}}
\providecommand{\system}{\overset{\mathcal{H}}{ \longleftrightarrow}}
	%\newcommand{\solution}[2]{\textbf{Solution:}{#1}}
\newcommand{\solution}{\noindent \textbf{Solution: }}
\newcommand{\cosec}{\,\text{cosec}\,}
\providecommand{\dec}[2]{\ensuremath{\overset{#1}{\underset{#2}{\gtrless}}}}
\newcommand{\myvec}[1]{\ensuremath{\begin{pmatrix}#1\end{pmatrix}}}
\newcommand{\mydet}[1]{\ensuremath{\begin{vmatrix}#1\end{vmatrix}}}

\numberwithin{equation}{subsection}

\makeatletter
\@addtoreset{figure}{problem}
\makeatother

\let\StandardTheFigure\thefigure
\let\vec\mathbf

\renewcommand{\thefigure}{\theproblem}

\def\putbox#1#2#3{\makebox[0in][l]{\makebox[#1][l]{}\raisebox{\baselineskip}[0in][0in]{\raisebox{#2}[0in][0in]{#3}}}}
     \def\rightbox#1{\makebox[0in][r]{#1}}
     \def\centbox#1{\makebox[0in]{#1}}
     \def\topbox#1{\raisebox{-\baselineskip}[0in][0in]{#1}}
     \def\midbox#1{\raisebox{-0.5\baselineskip}[0in][0in]{#1}}
\vspace{3cm}
\begin{center}
\huge Assignment 9\\
\large Surbhi Agarwal\\
\end{center}
\renewcommand{\thefigure}{\theenumi}
\renewcommand{\thetable}{\theenumi}

\begin{abstract}
This document deals with properties of subspaces of a finite dimensional vector space.
\end{abstract}

\section{Problem}
Let $W_1$ and $W_2$ be subspaces of a finite-dimensional vector space $\mathbb V$. Prove that
\begin{enumerate}
    \item $(W_1 + W_2)^0 = W_1^0 \cap W_2^0$
    \item $(W_1 \cap W_2)^0 = W_1^0 + W_2^0$
\end{enumerate}
\section{Solution}
\renewcommand{\thetable}{1}
\begin{longtable}{|l|l|}
    \hline
        & \\
        Given &  $W_1$ and $W_2$ are subspaces of a finite dimensional vector space $\mathbb V$\\
        & \\
        \hline
        & \\
        1. To prove & $(W_1 + W_2)^0 = W_1^0 \cap W_2^0$\\
        & \\
    \hline
        & \\
        Proof & Let $f \in (W_1 + W_2)^0$\\
        of & \\
        $(W_1 + W_2)^0$ & $\forall \vec{v} \in (W_1 + W_2)$\\
        $\subseteq$ & $f(\vec{v}) = 0$\\
        $(W_1^0 \cap W_2^0)$ & $\implies \forall \vec{w}_1 \in W_1, \vec{w}_2 \in W_2$\\
        & $\vec{w_1} + \vec{w_2} \in (W_1 + W_2)$\\
        & $\therefore f(\vec{w_1} + \vec{w_2}) = 0$\\
        & \\
        & $\implies$ When $\vec{w}_2 = 0$, then $\forall \vec{w}_1 \in W_1$\\
        & $f(\vec{w_1}) = 0$\\
        & $\therefore f \in W_1^0$\\
        &\\
        & And when $\vec{w_1} = 0, \forall \vec{w_2} \in W_2$\\
        & $f(\vec{w_2}) = 0$\\
        & $\therefore f \in W_2^0$\\
        & \\
        & $\because f \in W_1^0, f \in W_2^0$\\
        & $\implies f \in (W_1^0 \cap W_2^0)$\\
        & \\
        & $\therefore (W_1 + W_2)^0 \subseteq (W_1^0 \cap W_2^0)$\\
    \hline
        & \\
        Proof & Let $f \in (W_1^0 \cap W_2^0)$\\
        of & \\
        $(W_1^0 \cap W_2^0)$ & $\implies f \in W_1^0, f \in W_2^0$\\
        $\subseteq$ & $\implies \forall \vec{w}_1 \in W_1, \vec{w}_2 \in W_2$\\
        $(W_1 + W_2)^0$ & $f(\vec{w}_1) = 0$ and $f(\vec{w}_2) = 0$\\
        & \\
        & $\forall \vec{v} \in (W_1+W_2),$\\
    \hline
        & $\vec{v} = \vec{w}_1 + \vec{w}_2$\\
        & $\implies f(\vec{v}) = f(\vec{w}_1 + \vec{w}_2)$\\
        & $\implies f(\vec{v}) = f(\vec{w}_1) + f(\vec{w}_2)$\\
        & $\implies f(\vec{v}) = 0$\\
        & $\implies f \in (W_1 + W_2)^0$\\
        & \\
        & $\therefore (W_1^0 \cap W_2^0) \subseteq (W_1 + W_2)^0$\\
        & \\
    \hline
        & \\
        & Hence, $(W_1 + W_2)^0 = (W_1^0 \cap W_2^0)$\\
        & \\
    \hline
        & \\
        2. To prove & $(W_1 \cap W_2)^0 = W_1^0 + W_2^0$\\
        & \\
    \hline
        & \\
        Proof & Let $f \in (W_1^0 + W_2^0)$\\
        of & for some $f_1 \in W_1^0, f_2 \in W_2^0$,\\
        $(W_1^0 + W_2^0)$ & $f = f_1 + f_2$\\
        $\subseteq$ & \\
        $(W_1 \cap W_2)^0$ & Now, for $\vec{v} \in (W_1 \cap W_2)$\\
        & $f(\vec{v}) = (f_1 + f_2)(\vec{v})$\\
        & $\implies f(\vec{v}) = f_1(\vec{v}) + f_2(\vec{v})$\\
        & \\
        & $\because \vec{v} \in (W_1 \cap W_2)$\\
        & $\implies \vec{v} \in W_1$, and $\vec{v} \in W_2$\\
        & So, $f_1(\vec{v}) = 0$, and $f_2(\vec{v}) = 0$\\
        & \\
        & $\implies f(\vec{v}) = 0 + 0 = 0$\\
        & $\implies f \in (W_1 \cap W_2)^0$\\
        & \\
        & $\therefore (W_1^0 + W_2^0) \subseteq (W_1 \cap W_2)^0$\\
        & \\
    \hline
        & \\
        Proof & Let $f \in (W_1 \cap W_2)^0$\\
        of & \\
        $(W_1 \cap W_2)^0$ & Assuming\\
        $\subseteq$ & Basis of $W_1$ as $\{ \vec{\alpha}_1,\ldots,\vec{\alpha_k}, \vec{\beta}_1,\ldots,\vec{\beta}_l \}$\\
        $(W_1^0 + W_2^0)$ & Basis of $W_2$ as $\{ \vec{\alpha}_1,\ldots,\vec{\alpha_k}, \vec{\gamma}_1,\ldots,\vec{\gamma}_m \}$\\
        & \\
        & $\therefore$ Basis of $(W_1 \cap W_2)$ is $\{ \vec{\alpha}_1,\ldots,\vec{\alpha_k} \}$\\
        & and Basis of $W_1 + W_2$ is $\{ \vec{\alpha}_1,\ldots,\vec{\alpha_k}, \vec{\beta}_1,\ldots,\vec{\beta}_l, \vec{\gamma}_1,\ldots,\vec{\gamma}_m\}$\\
        & \\
        & Now, for $\vec{v} \in (W_1 + W_2)$\\
        & $\vec{v} = \sum_{i=1}^{k} x_i\alpha_i + \sum_{i=1}^{l} y_i\beta_i + \sum_{i=1}^{m} z_i\gamma_i$\\
        & $f(\vec{v}) = \sum_{i=1}^{k} a_ix_i + \sum_{i=1}^{l} b_iy_i + \sum_{i=1}^{m} c_iz_i$\\
        & \\
        & $\forall \vec{v} \in (W_1 \cap W_2)$\\
        & $\vec{v} = \sum_{i=1}^{k} x_i\alpha_i$\\
    \hline
        & $f(\vec{v}) = \sum_{i=1}^{k} a_ix_i$\\
        & But since $f \in (W_1 \cap W_2)^0$, thus $f(\vec{v}) = 0$\\
        & $\therefore a_1 = a_2 = \ldots = a_k = 0$\\
        & \\
        & So, we can now write\\
        & $f(\vec{v}) = \sum_{i=1}^{l} b_iy_i + \sum_{i=1}^{m} c_iz_i$\\
        & \\
        & Now, $\forall \vec{v} \in W_1$,\\
        & $\vec{v} = \sum_{i=1}^{k} x_i\alpha_i + \sum_{i=1}^{l} y_i\beta_i$\\
        & $\implies f_1(\vec{v}) = \sum_{i=1}^{k} a_ix_i + \sum_{i=1}^{l} b_iy_i$\\
        & Comparing this to the original equation, we can say $c_i = 0$\\
        & \\
        & And $\forall \vec{v} \in W_2$,\\
        & $\vec{v} = \sum_{i=1}^{k} x_i\alpha_i + \sum_{i=1}^{m} z_i\gamma_i$\\
        & $\implies f_2(\vec{v}) = \sum_{i=1}^{k} a_ix_i + \sum_{i=1}^{m} c_iz_i$\\
        & Comparing this to the original equation, we can say $b_i = 0$\\
        & \\
        & $f = f_1 + f_2$\\
        & \\
        & $\because a_i = b_i = c_i = 0$,\\
        & $f_1(\vec{v}) = 0$\\
        & $\implies f_1 \in W_1^0$\\
        & Also, $f_2(\vec{v}) = 0$\\
        & $\implies f_2 \in W_2^0$\\
        & So, $f_1 + f_2 \in (W_1^0 + W_2^0)$\\
        & $\implies f \in (W_1^0 + W_2^0)$\\
        & \\
        & $\therefore (W_1 \cap W_2)^0 \subseteq (W_1^0 + W_2^0)$\\
        & \\
    \hline
        & \\
        & Hence, $(W_1 \cap W_2)^0 = (W_1^0 + W_2^0)$\\
        & \\
    \hline
        & \\
        Verification & 1. Since the annihilator $(W_1 + W_2)^0$ is a complement of $(W_1 + W_2)$,\\
        & Using Demorgan's Laws of the complement of union of two sets is\\
        & the intersection of their complements, it is verified\\
        & $(W_1 + W_2)^0 = W_1^0 \cap W_2^0$\\
        &\\
        & 2. And using De Morgan's laws of the complement of intersection of two sets is\\
        & the union of their complements, it is verified\\
        & $(W_1 \cap W_2)^0 = W_1^0 + W_2^0$\\
        & \\
    \hline
    \caption{Proving properties of vectorspaces and subspaces}
    \label{tab:proof}
\end{longtable}
\end{document}