\documentclass[journal,12pt,twocolumn]{IEEEtran}

\usepackage{setspace}
\usepackage{gensymb}


\singlespacing

\usepackage[cmex10]{amsmath}
%\usepackage{amsthm}
%\interdisplaylinepenalty=2500
%\savesymbol{iint}
%\usepackage{txfonts}
%\restoresymbol{TXF}{iint}
%\usepackage{wasysym}
\usepackage{amsthm}

\usepackage{mathrsfs}
\usepackage{txfonts}
\usepackage{stfloats}
\usepackage{bm}
\usepackage{cite}
\usepackage{cases}
\usepackage{subfig}

\usepackage{longtable}
\usepackage{multirow}

\usepackage{enumitem}
\usepackage{mathtools}
\usepackage{steinmetz}
\usepackage{tikz}
\usepackage{circuitikz}
\usepackage{verbatim}
\usepackage{tfrupee}
\usepackage[breaklinks=true]{hyperref}

\usepackage{tkz-euclide} %loads TikZ and tkz-base

\usetikzlibrary{calc,math}
\usepackage{listings}
    \usepackage{color}                                          
    \usepackage{array}                                          
    \usepackage{longtable}                                      
    \usepackage{calc}                                           
    \usepackage{multirow}                                       
    \usepackage{hhline}                                         
    \usepackage{ifthen}
    \usepackage{lscape}     
\usepackage{multicol}
\usepackage{chngcntr}

\DeclareMathOperator*{\Res}{Res}

\renewcommand\thesection{\arabic{section}}
\renewcommand\thesubsection{\thesection.\arabic{subsection}}
\renewcommand\thesubsubsection{\thesubsection.\arabic{subsubsection}}

\renewcommand\thesectiondis{\arabic{section}}
\renewcommand\thesubsectiondis{\thesectiondis.\arabic{subsection}}
\renewcommand\thesubsubsectiondis{\thesubsectiondis.\arabic{subsubsection}}

\hyphenation{op-tical net-works semi-conduc-tor}
\def\inputGnumericTable{}                                 %%

\lstset{
%language=C,
frame=single, 
breaklines=true,
columns=fullflexible
}

\begin{document}

\newtheorem{theorem}{Theorem}[section]
\newtheorem{problem}{Problem}
\newtheorem{proposition}{Proposition}[section]
\newtheorem{lemma}{Lemma}[section]
\newtheorem{corollary}[theorem]{Corollary}
\newtheorem{example}{Example}[section]
\newtheorem{definition}[problem]{Definition}

\newcommand{\BEQA}{\begin{eqnarray}}
\newcommand{\EEQA}{\end{eqnarray}}
\newcommand{\define}{\stackrel{\triangle}{=}}

\bibliographystyle{IEEEtran}

\providecommand{\mbf}{\mathbf}
\providecommand{\pr}[1]{\ensuremath{\Pr\left(#1\right)}}
\providecommand{\qfunc}[1]{\ensuremath{Q\left(#1\right)}}
\providecommand{\sbrak}[1]{\ensuremath{{}\left[#1\right]}}
\providecommand{\lsbrak}[1]{\ensuremath{{}\left[#1\right.}}
\providecommand{\rsbrak}[1]{\ensuremath{{}\left.#1\right]}}
\providecommand{\brak}[1]{\ensuremath{\left(#1\right)}}
\providecommand{\lbrak}[1]{\ensuremath{\left(#1\right.}}
\providecommand{\rbrak}[1]{\ensuremath{\left.#1\right)}}
\providecommand{\cbrak}[1]{\ensuremath{\left\{#1\right\}}}
\providecommand{\lcbrak}[1]{\ensuremath{\left\{#1\right.}}
\providecommand{\rcbrak}[1]{\ensuremath{\left.#1\right\}}}
\theoremstyle{remark}
\newtheorem{rem}{Remark}
\newcommand{\sgn}{\mathop{\mathrm{sgn}}}
\providecommand{\abs}[1]{\left\vert#1\right\vert}
\providecommand{\res}[1]{\Res\displaylimits_{#1}} 
\providecommand{\norm}[1]{\left\lVert#1\right\rVert}
%\providecommand{\norm}[1]{\lVert#1\rVert}
\providecommand{\mtx}[1]{\mathbf{#1}}
\providecommand{\mean}[1]{E\left[ #1 \right]}
\providecommand{\fourier}{\overset{\mathcal{F}}{ \rightleftharpoons}}
%\providecommand{\hilbert}{\overset{\mathcal{H}}{ \rightleftharpoons}}
\providecommand{\system}{\overset{\mathcal{H}}{ \longleftrightarrow}}
	%\newcommand{\solution}[2]{\textbf{Solution:}{#1}}
\newcommand{\solution}{\noindent \textbf{Solution: }}
\newcommand{\cosec}{\,\text{cosec}\,}
\providecommand{\dec}[2]{\ensuremath{\overset{#1}{\underset{#2}{\gtrless}}}}
\newcommand{\myvec}[1]{\ensuremath{\begin{pmatrix}#1\end{pmatrix}}}
\newcommand{\mydet}[1]{\ensuremath{\begin{vmatrix}#1\end{vmatrix}}}

\numberwithin{equation}{subsection}

\makeatletter
\@addtoreset{figure}{problem}
\makeatother

\let\StandardTheFigure\thefigure
\let\vec\mathbf

\renewcommand{\thefigure}{\theproblem}

\def\putbox#1#2#3{\makebox[0in][l]{\makebox[#1][l]{}\raisebox{\baselineskip}[0in][0in]{\raisebox{#2}[0in][0in]{#3}}}}
     \def\rightbox#1{\makebox[0in][r]{#1}}
     \def\centbox#1{\makebox[0in]{#1}}
     \def\topbox#1{\raisebox{-\baselineskip}[0in][0in]{#1}}
     \def\midbox#1{\raisebox{-0.5\baselineskip}[0in][0in]{#1}}
\vspace{3cm}

\title{Assignment 1}
\author{Surbhi Agarwal}

\maketitle

\newpage

%\tableofcontents

\bigskip

\renewcommand{\thefigure}{\theenumi}
\renewcommand{\thetable}{\theenumi}

\begin{abstract}
This document solves a problem from Points and Vectors, where it finds the angle between a given Force and displacement vector
\end{abstract}

Download all python codes from 
\begin{lstlisting}
https://github.com/surbhi0912/EE5609/tree/master/codes
\end{lstlisting}
%
and latex-tikz codes from 
%
\begin{lstlisting}
https://github.com/surbhi0912/EE5609
\end{lstlisting}
%
\section{Problem}
Find the angle between the force $\vec{F} = \myvec{3 \\ 4\\ -5}$ and displacement $\vec{d} = \myvec{5 \\ 4\\ 3}$
\section{Explanation}
Let the angle between $\vec{F}$ and $\vec{d}$ = $\theta$
Then,
\begin{align}\label{eqn1}
    \cos(\theta) = \dfrac{\vec{F}^T \vec{d}}{\norm{\vec{F}} \norm{\vec{d}}}
\end{align}

where $\vec{F}^T\vec{d}$ is scalar product of vectors $\vec{F}$ and $\vec{d}$

And, $\norm{\vec{F}}$ and $\norm{\vec{d}}$ are their respective magnitudes
\section{Solution}
So,
\begin{align}\label{eqn2}
    \vec{F}^T\vec{d} & = \myvec{3 \\ 4 \\ -5}^T \myvec{5 \\ 4\\ 3}
\end{align}
\begin{align}\label{eqn3}
    \implies\vec{F}^T\vec{d} & = \myvec{3 & 4 & -5} \myvec{5 \\ 4 \\ 3} \\& = 16
\end{align}
\begin{align}\label{eqn4}
    \norm{\vec{F}} = \sqrt{3^2 + 4^2 + (-5)^2} = 5\sqrt{2}
\end{align}
\begin{align}\label{eqn5}
    \norm{\vec{d}} = \sqrt{5^2 + 4^2 + 3^2} = 5\sqrt{2}
\end{align}

Substituting these values in Equation \ref{eqn1},
\begin{align}
    \cos(\theta) & = \dfrac{16}{(5\sqrt{2})(5\sqrt{2})} \\& = \dfrac{8}{25}
\end{align}
\begin{align}
    \implies\theta = \arccos \left({\dfrac{8}{25}} \right)
\end{align}
\begin{align}
    \implies\theta \approx \ 71.3\degree
\end{align}

Thus, angle between $\vec{F}$ and $\vec{d}$ $\approx 71.3\degree$
\end{document}