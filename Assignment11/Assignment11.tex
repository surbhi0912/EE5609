\documentclass[journal,12pt]{IEEEtran}

\usepackage{setspace}
\usepackage{gensymb}


\singlespacing

\usepackage[cmex10]{amsmath}
%\usepackage{amsthm}
%\interdisplaylinepenalty=2500
%\savesymbol{iint}
%\usepackage{txfonts}
%\restoresymbol{TXF}{iint}
%\usepackage{wasysym}
\usepackage{amsthm}

\usepackage{mathrsfs}
\usepackage{txfonts}
\usepackage{stfloats}
\usepackage{bm}
\usepackage{cite}
\usepackage{cases}
\usepackage{subfig}

\usepackage{longtable}
\usepackage{multirow}

\usepackage{enumitem}
\usepackage{mathtools}
\usepackage{steinmetz}
\usepackage{tikz}
\usepackage{circuitikz}
\usepackage{verbatim}
\usepackage{tfrupee}
\usepackage[breaklinks=true]{hyperref}

\usepackage{tkz-euclide} %loads TikZ and tkz-base

\usetikzlibrary{calc,math}
\usepackage{listings}
    \usepackage{color}                                          
    \usepackage{array}                                          
    \usepackage{longtable}                                      
    \usepackage{calc}                                           
    \usepackage{multirow}                                       
    \usepackage{hhline}                                         
    \usepackage{ifthen}
    \usepackage{lscape}     
\usepackage{multicol}
\usepackage{chngcntr}

\DeclareMathOperator*{\Res}{Res}

\renewcommand\thesection{\arabic{section}}
\renewcommand\thesubsection{\thesection.\arabic{subsection}}
\renewcommand\thesubsubsection{\thesubsection.\arabic{subsubsection}}

\renewcommand\thesectiondis{\arabic{section}}
\renewcommand\thesubsectiondis{\thesectiondis.\arabic{subsection}}
\renewcommand\thesubsubsectiondis{\thesubsectiondis.\arabic{subsubsection}}

\hyphenation{op-tical net-works semi-conduc-tor}
\def\inputGnumericTable{}                                 %%

\lstset{
%language=C,
frame=single, 
breaklines=true,
columns=fullflexible
}

\begin{document}

\newtheorem{theorem}{Theorem}[section]
\newtheorem{problem}{Problem}
\newtheorem{proposition}{Proposition}[section]
\newtheorem{lemma}{Lemma}[section]
\newtheorem{corollary}[theorem]{Corollary}
\newtheorem{example}{Example}[section]
\newtheorem{definition}[problem]{Definition}

\newcommand{\BEQA}{\begin{eqnarray}}
\newcommand{\EEQA}{\end{eqnarray}}
\newcommand{\define}{\stackrel{\triangle}{=}}

\bibliographystyle{IEEEtran}

\providecommand{\mbf}{\mathbf}
\providecommand{\pr}[1]{\ensuremath{\Pr\left(#1\right)}}
\providecommand{\qfunc}[1]{\ensuremath{Q\left(#1\right)}}
\providecommand{\sbrak}[1]{\ensuremath{{}\left[#1\right]}}
\providecommand{\lsbrak}[1]{\ensuremath{{}\left[#1\right.}}
\providecommand{\rsbrak}[1]{\ensuremath{{}\left.#1\right]}}
\providecommand{\brak}[1]{\ensuremath{\left(#1\right)}}
\providecommand{\lbrak}[1]{\ensuremath{\left(#1\right.}}
\providecommand{\rbrak}[1]{\ensuremath{\left.#1\right)}}
\providecommand{\cbrak}[1]{\ensuremath{\left\{#1\right\}}}
\providecommand{\lcbrak}[1]{\ensuremath{\left\{#1\right.}}
\providecommand{\rcbrak}[1]{\ensuremath{\left.#1\right\}}}
\theoremstyle{remark}
\newtheorem{rem}{Remark}
\newcommand{\sgn}{\mathop{\mathrm{sgn}}}
\providecommand{\abs}[1]{\left\vert#1\right\vert}
\providecommand{\res}[1]{\Res\displaylimits_{#1}} 
\providecommand{\norm}[1]{\left\lVert#1\right\rVert}
%\providecommand{\norm}[1]{\lVert#1\rVert}
\providecommand{\mtx}[1]{\mathbf{#1}}
\providecommand{\mean}[1]{E\left[ #1 \right]}
\providecommand{\fourier}{\overset{\mathcal{F}}{ \rightleftharpoons}}
%\providecommand{\hilbert}{\overset{\mathcal{H}}{ \rightleftharpoons}}
\providecommand{\system}{\overset{\mathcal{H}}{ \longleftrightarrow}}
	%\newcommand{\solution}[2]{\textbf{Solution:}{#1}}
\newcommand{\solution}{\noindent \textbf{Solution: }}
\newcommand{\cosec}{\,\text{cosec}\,}
\providecommand{\dec}[2]{\ensuremath{\overset{#1}{\underset{#2}{\gtrless}}}}
\newcommand{\myvec}[1]{\ensuremath{\begin{pmatrix}#1\end{pmatrix}}}
\newcommand{\mydet}[1]{\ensuremath{\begin{vmatrix}#1\end{vmatrix}}}

\numberwithin{equation}{subsection}

\makeatletter
\@addtoreset{figure}{problem}
\makeatother

\let\StandardTheFigure\thefigure
\let\vec\mathbf

\renewcommand{\thefigure}{\theproblem}

\def\putbox#1#2#3{\makebox[0in][l]{\makebox[#1][l]{}\raisebox{\baselineskip}[0in][0in]{\raisebox{#2}[0in][0in]{#3}}}}
     \def\rightbox#1{\makebox[0in][r]{#1}}
     \def\centbox#1{\makebox[0in]{#1}}
     \def\topbox#1{\raisebox{-\baselineskip}[0in][0in]{#1}}
     \def\midbox#1{\raisebox{-0.5\baselineskip}[0in][0in]{#1}}
\vspace{3cm}
\begin{center}
\huge Assignment 11\\
\large Surbhi Agarwal\\
\end{center}
\renewcommand{\thefigure}{\theenumi}
\renewcommand{\thetable}{\theenumi}

\begin{abstract}
This document illustrates linear transformation matrices and 
\end{abstract}

\section{Problem}
Let $\vec{S}: \mathbb R^n \rightarrow \mathbb R^n$ be given by $\vec{S}(\vec{v}) = \alpha\vec{v}$, for a fixed $\alpha \in \mathbb R, \alpha \neq 0$. Let $\vec{T}: \mathbb R^n \rightarrow \mathbb R^n$ be a linear transformation such that $\vec{B} = \{ \vec{v}_1,\ldots,\vec{v}_n \}$ is a set of linearly independent eigenvectors of $\vec{T}$. Then
\begin{enumerate}
    \item The matrix of $\vec{T}$ with respect to $\vec{B}$ is diagonal
    \item The matrix of $(\vec{T}-\vec{S})$ with respect to $\vec{B}$ is diagonal
    \item The matrix of $\vec{T}$ with respect to $\vec{B}$ is not necessarily diagonal, but is upper triangular
    \item The matrix of $\vec{T}$ with respect to $\vec{B}$ is diagonal but the matrix of $(\vec{T}-\vec{S})$ with respect to $\vec{B}$ is not diagonal.
\end{enumerate}
\section{Solution}
Given that $\vec{T}: \mathbb R^n \rightarrow \mathbb R^n$ be a linear transformation and B represents a set of linearly independent eigenvectors of $\vec{T}$ given as follows
\begin{align}
    \vec{B} = \{\vec{v}_1,\ldots,\vec{v}_n\}
\end{align}
So,
\begin{align}
    \vec{T}\vec{v}_i = \lambda_i\vec{v}_i
\end{align}
where $\lambda_i$ represents the eigenvalue $\lambda_i$ corresponding to $\vec{v}_i$. Hence, the matrix $\vec{T}$ with respect to $\vec{B}$ can be represented as
\begin{align}
    \vec{T} = \myvec{\lambda_1 & 0 & \dots & 0\\ 0 & \lambda_2 & \dots & 0\\ \vdots & \ddots &  & \\ 0 & \dots & 0 & \lambda_n}\label{eq:T}
\end{align}
And,
\begin{align}
    (\vec{T}-\vec{S})\vec{v}_i & = \vec{T}(\vec{v}_i) - \vec{S}(\vec{v}_i)\\& = \lambda_i\vec{v}_i - \alpha\vec{v}_i \\ & = (\lambda_i - \alpha)\vec{v}_i
\end{align}
Hence, matrix of $\vec{T}-\vec{S}$ with respect to $\vec{B}$ can be represented as
\begin{align}
    \vec{T}-\vec{S} = \myvec{\lambda_1-\alpha & 0 & \dots & 0\\ 0 & \lambda_2-\alpha & \dots & 0\\ \vdots & \ddots &  & \\ 0 & \dots & 0 & \lambda_n-\alpha}\label{eq:T-S}
\end{align}
\renewcommand{\thetable}{1}
\begin{longtable}{|l|l|}
    \hline
        & \\
        1. The matrix of $\vec{T}$ & True, as seen\\
        w.r.t to $\vec{B}$ is diagonal & from \eqref{eq:T}\\
        & \\
    \hline
        & \\
        2. The matrix of $(\vec{T}-\vec{S})$ & True, as seen\\
        w.r.t $\vec{B}$ is diagonal & from \eqref{eq:T-S}\\
        & \\
    \hline
        & \\
        3. The matrix of $\vec{T}$ with respect & False, as\\
        to $\vec{B}$ is not necessarily diagonal & already proved\\
        but is upper triangular & $\vec{T}$ is diagonal\\
        & \\
    \hline
        & \\
        4. The matrix of $\vec{T}$ with respect to $\vec{B}$ & False, as\\
        is diagonal but the matrix of $(\vec{T}-\vec{S})$ & already proved\\
        with respect to $\vec{B}$ is not diagonal & $\vec{T}-\vec{S}$ is diagonal\\
        & \\
    \hline
    \caption{Verifying the given options}
    \label{tab:proof}
\end{longtable}
\end{document}