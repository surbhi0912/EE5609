\documentclass[journal,12pt,twocolumn]{IEEEtran}

\usepackage{setspace}
\usepackage{gensymb}


\singlespacing

\usepackage[cmex10]{amsmath}
%\usepackage{amsthm}
%\interdisplaylinepenalty=2500
%\savesymbol{iint}
%\usepackage{txfonts}
%\restoresymbol{TXF}{iint}
%\usepackage{wasysym}
\usepackage{amsthm}

\usepackage{mathrsfs}
\usepackage{txfonts}
\usepackage{stfloats}
\usepackage{bm}
\usepackage{cite}
\usepackage{cases}
\usepackage{subfig}

\usepackage{longtable}
\usepackage{multirow}

\usepackage{enumitem}
\usepackage{mathtools}
\usepackage{steinmetz}
\usepackage{tikz}
\usepackage{circuitikz}
\usepackage{verbatim}
\usepackage{tfrupee}
\usepackage[breaklinks=true]{hyperref}

\usepackage{tkz-euclide} %loads TikZ and tkz-base

\usetikzlibrary{calc,math}
\usepackage{listings}
    \usepackage{color}                                          
    \usepackage{array}                                          
    \usepackage{longtable}                                      
    \usepackage{calc}                                           
    \usepackage{multirow}                                       
    \usepackage{hhline}                                         
    \usepackage{ifthen}
    \usepackage{lscape}     
\usepackage{multicol}
\usepackage{chngcntr}

\DeclareMathOperator*{\Res}{Res}

\renewcommand\thesection{\arabic{section}}
\renewcommand\thesubsection{\thesection.\arabic{subsection}}
\renewcommand\thesubsubsection{\thesubsection.\arabic{subsubsection}}

\renewcommand\thesectiondis{\arabic{section}}
\renewcommand\thesubsectiondis{\thesectiondis.\arabic{subsection}}
\renewcommand\thesubsubsectiondis{\thesubsectiondis.\arabic{subsubsection}}

\hyphenation{op-tical net-works semi-conduc-tor}
\def\inputGnumericTable{}                                 %%

\lstset{
%language=C,
frame=single, 
breaklines=true,
columns=fullflexible
}

\begin{document}

\newtheorem{theorem}{Theorem}[section]
\newtheorem{problem}{Problem}
\newtheorem{proposition}{Proposition}[section]
\newtheorem{lemma}{Lemma}[section]
\newtheorem{corollary}[theorem]{Corollary}
\newtheorem{example}{Example}[section]
\newtheorem{definition}[problem]{Definition}

\newcommand{\BEQA}{\begin{eqnarray}}
\newcommand{\EEQA}{\end{eqnarray}}
\newcommand{\define}{\stackrel{\triangle}{=}}

\bibliographystyle{IEEEtran}

\providecommand{\mbf}{\mathbf}
\providecommand{\pr}[1]{\ensuremath{\Pr\left(#1\right)}}
\providecommand{\qfunc}[1]{\ensuremath{Q\left(#1\right)}}
\providecommand{\sbrak}[1]{\ensuremath{{}\left[#1\right]}}
\providecommand{\lsbrak}[1]{\ensuremath{{}\left[#1\right.}}
\providecommand{\rsbrak}[1]{\ensuremath{{}\left.#1\right]}}
\providecommand{\brak}[1]{\ensuremath{\left(#1\right)}}
\providecommand{\lbrak}[1]{\ensuremath{\left(#1\right.}}
\providecommand{\rbrak}[1]{\ensuremath{\left.#1\right)}}
\providecommand{\cbrak}[1]{\ensuremath{\left\{#1\right\}}}
\providecommand{\lcbrak}[1]{\ensuremath{\left\{#1\right.}}
\providecommand{\rcbrak}[1]{\ensuremath{\left.#1\right\}}}
\theoremstyle{remark}
\newtheorem{rem}{Remark}
\newcommand{\sgn}{\mathop{\mathrm{sgn}}}
\providecommand{\abs}[1]{\left\vert#1\right\vert}
\providecommand{\res}[1]{\Res\displaylimits_{#1}} 
\providecommand{\norm}[1]{\left\lVert#1\right\rVert}
%\providecommand{\norm}[1]{\lVert#1\rVert}
\providecommand{\mtx}[1]{\mathbf{#1}}
\providecommand{\mean}[1]{E\left[ #1 \right]}
\providecommand{\fourier}{\overset{\mathcal{F}}{ \rightleftharpoons}}
%\providecommand{\hilbert}{\overset{\mathcal{H}}{ \rightleftharpoons}}
\providecommand{\system}{\overset{\mathcal{H}}{ \longleftrightarrow}}
	%\newcommand{\solution}[2]{\textbf{Solution:}{#1}}
\newcommand{\solution}{\noindent \textbf{Solution: }}
\newcommand{\cosec}{\,\text{cosec}\,}
\providecommand{\dec}[2]{\ensuremath{\overset{#1}{\underset{#2}{\gtrless}}}}
\newcommand{\myvec}[1]{\ensuremath{\begin{pmatrix}#1\end{pmatrix}}}
\newcommand{\mydet}[1]{\ensuremath{\begin{vmatrix}#1\end{vmatrix}}}

\numberwithin{equation}{subsection}

\makeatletter
\@addtoreset{figure}{problem}
\makeatother

\let\StandardTheFigure\thefigure
\let\vec\mathbf

\renewcommand{\thefigure}{\theproblem}

\def\putbox#1#2#3{\makebox[0in][l]{\makebox[#1][l]{}\raisebox{\baselineskip}[0in][0in]{\raisebox{#2}[0in][0in]{#3}}}}
     \def\rightbox#1{\makebox[0in][r]{#1}}
     \def\centbox#1{\makebox[0in]{#1}}
     \def\topbox#1{\raisebox{-\baselineskip}[0in][0in]{#1}}
     \def\midbox#1{\raisebox{-0.5\baselineskip}[0in][0in]{#1}}
\vspace{3cm}

\title{Assignment 6}
\author{Surbhi Agarwal}

\maketitle

\newpage

%\tableofcontents

\bigskip

\renewcommand{\thefigure}{\theenumi}
\renewcommand{\thetable}{\theenumi}

\begin{abstract}
This document deals with QR decomposition and Singlular Value Decomposition
\end{abstract}

Download all python codes from 
%
\begin{lstlisting}
https://github.com/surbhi0912/EE5609/
\end{lstlisting}
%
and latex-tikz codes from 
%
\begin{lstlisting}
https://github.com/surbhi0912/EE5609/
\end{lstlisting}
%
\section{Problem}
1. Find the QR decomposition of $\vec{V} = \myvec{16 & 12\\12 & 9}$

2. Find the vertex of a parabola
\begin{align}\nonumber
    (4x+3y+15)^2 = 5(3x-4y)
\end{align}
using SVD and verify solution using least squares.
\section{Solution}
\subsection{QR decomposition of $\vec{V}$}
Let the column vectors of $\vec{V}$ be $\alpha$ and $\beta$:
\begin{align}
    \alpha=\myvec{16 \\ 12}\\
    \beta=\myvec{12 \\9}
\end{align}
We can express
\begin{align}
    \alpha = k_1\vec{u}_1 \label{eq:alpha}\\
    \beta = r_1\vec{u}_1+k_2\vec{u}_2\label{eq:beta}
    \intertext{where}
    k_1 = \norm\alpha = \sqrt{16^2+12^2} =20 \\
    \vec{u}_1 = \frac{\alpha}{k_1} = \frac{1}{20}\myvec{16 \\ 12} = \myvec{\frac{4}{5} \\ \frac{3}{5}}\\
    r_1 = \frac{\vec{u}_1^T\beta}{\norm{\vec{u}_1}^2} = 15\\
    \vec{u}_2 = \frac{\beta - r_1\vec{u}_1}{\norm{\beta - r_1\vec{u}_1}} = \myvec{0\\0}\\
    k_2 = \vec{u}_2^T\beta = 0
\end{align}
From \eqref{eq:alpha} and \eqref{eq:beta},
\begin{align}
    \myvec{\alpha & \beta} = \myvec{\vec{u}_1 & \vec{u}_2}\myvec{k_1 & r_1\\0 & k_2}
    \intertext{where,}
    \vec{Q} = \myvec{\vec{u}_1 & \vec{u}_2}\\
    \vec{R} = \myvec{k_1 & r_1\\0 & k_2}
\end{align}
$\vec{R}$ should be an upper triangular matrix and $\vec{Q}$ an orthogonal matrix such that $\vec{Q}^T\vec{Q} = \vec{I}$

Here, we see that the second column vector of $\vec{Q}$ is zero since the column vectors of $\vec{V}$ are dependent. Therefore, we can effecively write $\vec{Q}$ as:
\begin{align}
    \vec{Q} = \myvec{\frac{4}{5} \\ \frac{3}{5}}
\end{align}
Verifying $\vec{Q}^T\vec{Q} = \vec{I}$
\begin{align}
    \vec{Q}^T\vec{Q} = \myvec{\frac{4}{5} & \frac{3}{5}}\myvec{\frac{4}{5} \\ \frac{3}{5}} = 1
\end{align}
Therefore, for the given matrix $\vec{V}$, we can write $\vec{Q}\vec{R}$ decomposition as the product of respective row and column vectors as,
\begin{align}
    \vec{V} = \vec{Q}\vec{R} = \myvec{\frac{4}{5} \\ \frac{3}{5}}\myvec{20 & 15}
\end{align}
\subsection{Singular Value Decomposition for finding Vertex}
The given equation can be rewritten as
\begin{align}\label{eq:quadraticparabola}
    16x^2+24xy+9y^2+105x+110y+225 = 0
\end{align}
Comparing this to the standard equation,
%\vec{x}^T\vec{V}\vec{x}+2\vec{u}^T\vec{x}+f = 0
\begin{align}
    \vec{V} = \vec{V}^T = \myvec{16 & 12\\12 & 9}, \quad \vec{u} = \myvec{\frac{105}{2} \\ 55}, \quad f = 225 \label{eq:Vufvals}
\end{align}
The characteristic equation of $\vec{V}$ is given as
\begin{align}
    \mydet{\lambda\vec{I}-\vec{V}} = 0\\
    \implies \mydet{\lambda-16 & -12 \\ -12 & \lambda-9} = 0\\
    \implies \lambda^2 -25\lambda = 0 \label{eq:lambdaeq}
\end{align}
The eigenvalues are the roots of the equation \eqref{eq:lambdaeq}, which are as follows :
\begin{align}
    \lambda_1 = 0, \quad \lambda_2 = 25 \label{eq:eigenval}
\end{align}
The eigen vector $\vec{p}$ is defined as, 
\begin{align}
    \vec{V}\vec{p} &= \lambda\vec{p}\\
    \implies(\lambda\vec{I}-\vec{V})\vec{p}&=0
\end{align}
For $\lambda_1=0$
\begin{align}
    (\lambda_1\vec{I}-\vec{V}) = \myvec{-16 & -12\\-12 & -9}\xleftrightarrow[R_2\leftarrow R_2-3R_1]{R_1\leftarrow \frac{1}{4}R_1}\myvec{-4 & -3\\0 & 0}
\end{align}
\begin{align}
    \implies\vec{p_1}&=\frac{1}{5}\myvec{-3\\4}\label{eq:p1val}
\end{align}
For $\lambda_2=25$
\begin{align}
    (\lambda_2\vec{I}-\vec{V}) = \myvec{9 & -12\\-12 & 16}\xleftrightarrow[R_2\leftarrow R_2+4R_1]{R_1\leftarrow \frac{1}{3}R_1}\myvec{3 & -4\\0 & 0}
\end{align}
\begin{align}
    \implies\vec{p_2}=\frac{1}{5}\myvec{4\\3}\label{eq:p2val}
\end{align}
So, using Eigenvalue decomposition, $\vec{P}^T\vec{V}\vec{P}=\vec{D}$, where
\begin{align}
    \vec{P} = \frac{1}{5}\myvec{-3 & 4\\4 & 3}\\
    \vec{D} = \myvec{\lambda_1 & 0\\0 & \lambda_2} = \myvec{0 & 0\\ 0 & 25}
\end{align}
Then, for the parabola
\begin{align}
    \text{focal length} = \abs{\frac{2\eta}{\lambda_2}} \label{eq:focallen}\\
    \eta = \vec{p}_1^T\vec{u} = \frac{25}{2}\label{eq:etaval}\\
    \intertext{Substituting values from \eqref{eq:etaval} and \eqref{eq:eigenval} in \eqref{eq:focallen}, we get}
    \text{focal length} = 1
\end{align}
The standard equation of the parabola is given by
\begin{align}
    \vec{y}^T\vec{D}\vec{y} = -2\eta\myvec{1 & 0}\vec{y}
\end{align}
And the vertex $\vec{c}$ is given by
\begin{align}
    \myvec{\vec{u}^T + 2\eta\vec{p}_1^T \\ \vec{V}}\vec{c} = \myvec{-f \\ 2\eta\vec{p}_1 - \vec{u}} \label{eq:c}
\end{align}
Substituting values from \eqref{eq:Vufvals},\eqref{eq:etaval},\eqref{eq:p1val} in \eqref{eq:c},
\begin{align}
    \myvec{\frac{75}{2} & 75\\16 & 12\\12 & 9}\vec{c} = \myvec{-225 \\ \frac{-135}{2} \\ -35}
\end{align}
This is of the form
\begin{align}
    \vec{A}\vec{c} = \vec{b}\label{eq:Acb}
\end{align}
To solve this, we perform Singular Value Decomposition of $\vec{A}$ as follows :
\begin{align}
    \vec{A} = \vec{U}\vec{S}\vec{V}^T\label{eq:Asvd}
\end{align}
where columns of $\vec{V}$ are eigen vectors of $\vec{A}^T\vec{A}$, columns of $\vec{U}$ are eigen vectors of $\vec{A}\vec{A}^T$ and $\vec{S}$ is the diagonal matrix of singular value of eigenvalues of $\vec{A}^T\vec{A}$.
Now, using \eqref{eq:Asvd} in \eqref{eq:Acb}, we get
\begin{align}
    \vec{U}\vec{S}\vec{V}^T\vec{c} = \vec{b}\\
    \implies \vec{c} = \vec{V}\vec{S}_+\vec{U}^T\vec{b}\label{eq:Ccalculation}
\end{align}
where $\vec{S}_+$ is the Moore-Penrose Pseduo-Inverse of $\vec{S}$. Now, we see
\begin{align}
    \vec{A}\vec{A}^T = \myvec{\dfrac{28125}{4} & 1500 & 1125 \\ 1500 & 400 & 300\\ 1125 & 300 & 225}\\
    \vec{A}^T\vec{A} = \myvec{\dfrac{7225}{4} & \dfrac{6225}{2} \\ \dfrac{6225}{2} & 5850}
\end{align}
Eigen values and vectors for $\vec{A}\vec{A}^T$
\begin{align}
    \mydet{\vec{A}\vec{A}^T - \lambda\vec{I}} = 0\\
    \implies \mydet{\dfrac{28125}{4} - \lambda & 1500 & 1125 \\ 1500 & 400 - \lambda & 300\\ 1125 & 300 & 225 - \lambda} = 0\\
    \implies -\lambda^3 + \frac{30625}{4}\lambda^2 - \frac{3515625}{4}\lambda = 0\label{eq:lambdaforAAT}
\end{align}
Solving \eqref{eq:lambdaforAAT}, we get
\begin{align}
    \lambda_1 = \frac{-625\sqrt{2257} + 30625}{8}\\
    \lambda_2 = \frac{625\sqrt{2257} + 30625}{8}\\
    \lambda_3 = 0
\end{align}
The normalized eigen vector corresponding to these eigen values is:
\begin{align}
    \vec{u}_1 = \myvec{\frac{-205 + 5\sqrt{2257}}{\sqrt{(205 - 5\sqrt{2257})^2 + 14400}}\\ \frac{-96}{\sqrt{(205 - 5\sqrt{2257})^2 + 14400}}\\\frac{-72}{\sqrt{(205 - 5\sqrt{2257})^2 + 14400}}}\\
    \vec{u}_2 = \myvec{\frac{205 + 5\sqrt{2257}}{\sqrt{(205 + 5\sqrt{2257})^2 + 14400}}\\ \frac{96}{\sqrt{(205 + 5\sqrt{2257})^2 + 14400}}\\\frac{72}{\sqrt{(205 + 5\sqrt{2257})^2 + 14400}}}\\
    \vec{u}_3 = \myvec{0 \\ \frac{-3}{5} \\ \frac{4}{5}}\\
    \intertext{Thus,}
    \vec{U} = \myvec{\frac{-205 + 5\sqrt{2257}}{\sqrt{(205 - 5\sqrt{2257})^2 + 14400}} & \frac{205 + 5\sqrt{2257}}{\sqrt{(205 + 5\sqrt{2257})^2 + 14400}} & 0\\ \frac{-96}{\sqrt{(205 - 5\sqrt{2257})^2 + 14400}} & \frac{96}{\sqrt{(205 + 5\sqrt{2257})^2 + 14400}} & \frac{-3}{5} \\ \frac{-72}{\sqrt{(205 - 5\sqrt{2257})^2 + 14400}} & \frac{72}{\sqrt{(205 + 5\sqrt{2257})^2 + 14400}} & \frac{4}{5}}\label{eq:Uval}
\end{align}
$\vec{S}$ corresponding to eigen-values $\lambda_1, \lambda_2, \lambda_3$ is:
%%taken by making nonzero eigenvalues as positive and then taking their square root and placing these singular values along the diagonal
\begin{align}
    \vec{S} = \myvec{\frac{\sqrt{-625\sqrt{2257} + 30625}}{2\sqrt{2}} & 0 \\ 0 & \frac{\sqrt{625\sqrt{2257} + 30625}}{2\sqrt{2}} \\ 0 & 0}\label{eq:Sval}
\end{align}
Eigen values and vectors for $\vec{A}^T\vec{A}$
\begin{align}
    \mydet{\vec{A}^T\vec{A} - \lambda\vec{I}} = 0\\
    \mydet{\dfrac{7225}{4} -\lambda & \dfrac{6225}{2} \\ \dfrac{6225}{2} & 5850 - \lambda} = 0\\
    \implies \lambda^2 - \frac{30625}{4}\lambda+\frac{3515625}{4} = 0\label{eq:lambdaforATA}
\end{align}
Solving \eqref{eq:lambdaforATA}, we get
\begin{align}
    \lambda_4 = \frac{-625\sqrt{2257} + 30625}{8}\\
    \lambda_5 = \frac{625\sqrt{2257} + 30625}{8}
\end{align}
The normalized eigen vector corresponding to these eigen values is :
\begin{align}
    \vec{v}_1 = \myvec{\frac{-25\sqrt{2257}-647}{\sqrt{(25\sqrt{2257}+647)^2 + 992016}} \\ \frac{996}{\sqrt{(25\sqrt{2257}+647)^2 + 992016}}}\\
    \vec{v}_2 = \myvec{\frac{25\sqrt{2257}-647}{\sqrt{(25\sqrt{2257}-647)^2 + 992016}} \\ \frac{996}{\sqrt{(25\sqrt{2257}-647)^2 + 992016}}}
    \intertext{Thus,}
    \vec{V} = \myvec{\frac{-25\sqrt{2257}-647}{\sqrt{(25\sqrt{2257}+647)^2 + 992016}} & \frac{25\sqrt{2257}-647}{\sqrt{(25\sqrt{2257}-647)^2 + 992016}} \\ \frac{996}{\sqrt{(25\sqrt{2257}+647)^2 + 992016}} & \frac{996}{\sqrt{(25\sqrt{2257}-647)^2 + 992016}}}\label{eq:Vval}
\end{align}
Using \eqref{eq:Uval}, \eqref{eq:Sval}, \eqref{eq:Vval} in \eqref{eq:Asvd}, we can write the Singular Value Decomposition of $\vec{A}$.
Now, the Moore-Penrose Pseudo Inverse of $\vec{S}$ is given by:
%%taking the reciprocal of the non zero diagonal elements of S and leaving all other zeros as it is, then taking a transpose of the matrix
\begin{align}
    \vec{S}_+ & = \myvec{\frac{2\sqrt{2}}{\sqrt{-625\sqrt{2257} + 30625}} & 0 \\ 0 & \frac{2\sqrt{2}}{\sqrt{625\sqrt{2257} + 30625}} \\ 0 & 0}^T\\&
    = \myvec{\frac{2\sqrt{2}}{\sqrt{-625\sqrt{2257} + 30625}} & 0 & 0\\ 0 & \frac{2\sqrt{2}}{\sqrt{625\sqrt{2257} + 30625}} & 0}
\end{align}\label{eq:S+val}
Now, using values from \eqref{eq:Vval}, \eqref{eq:S+val}, \eqref{eq:Uval} in \eqref{eq:Ccalculation}, we get:
\begin{align}
    \vec{c} = \myvec{-2.4 \\ -1.7999}
\end{align}
Verifying this solution using least squares,
\begin{align}
    \vec{A}^T\vec{A}\vec{c} = \vec{A}^T\vec{b}
    \intertext{Substituting values here, we get}
    \myvec{\dfrac{7225}{4} & \dfrac{6225}{2} \\ \dfrac{6225}{2} & 5850}\vec{c} = \myvec{\frac{-19875}{2} \\ -18000}
\end{align}
Solving the augmented matrix
\begin{align}
    \myvec{\frac{7225}{4} & \frac{6225}{2} & \frac{-19875}{2}\\ \frac{6225}{2} & 5850 & -18000}\xleftrightarrow{R_1\leftarrow \frac{4}{7225}R_1}\myvec{1 & \frac{498}{289} & \frac{-1590}{289}\\ \frac{6225}{2} & 5850 & -18000}\\
    \xleftrightarrow{R_2\leftarrow R_2 - \frac{6225}{2}R_1}\myvec{1 & \frac{498}{289} & \frac{-1590}{289}\\0 & \frac{140625}{289} & \frac{-253125}{289}}\\
    \xleftrightarrow{R_2\leftarrow \frac{289}{140625}R_2}\myvec{1 & \frac{498}{289} & \frac{-1590}{289}\\0 & 1 & \frac{-9}{5}}\\
    \xleftrightarrow{R_1\leftarrow R_1 - \frac{498}{289}R_2}\myvec{1 & 0 & \frac{-12}{5}\\0 & 1 & \frac{-9}{5}}
\end{align}
Therefore,
\begin{align}
    \vec{c} = \myvec{\frac{-12}{5} \\ \frac{-9}{5}} = \myvec{-2.4 \\ -1.8}
\end{align}
Hence, verified.
\end{document}