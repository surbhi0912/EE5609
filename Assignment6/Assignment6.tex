\documentclass[journal,12pt,twocolumn]{IEEEtran}

\usepackage{setspace}
\usepackage{gensymb}


\singlespacing

\usepackage[cmex10]{amsmath}
%\usepackage{amsthm}
%\interdisplaylinepenalty=2500
%\savesymbol{iint}
%\usepackage{txfonts}
%\restoresymbol{TXF}{iint}
%\usepackage{wasysym}
\usepackage{amsthm}

\usepackage{mathrsfs}
\usepackage{txfonts}
\usepackage{stfloats}
\usepackage{bm}
\usepackage{cite}
\usepackage{cases}
\usepackage{subfig}

\usepackage{longtable}
\usepackage{multirow}

\usepackage{enumitem}
\usepackage{mathtools}
\usepackage{steinmetz}
\usepackage{tikz}
\usepackage{circuitikz}
\usepackage{verbatim}
\usepackage{tfrupee}
\usepackage[breaklinks=true]{hyperref}

\usepackage{tkz-euclide} %loads TikZ and tkz-base

\usetikzlibrary{calc,math}
\usepackage{listings}
    \usepackage{color}                                          
    \usepackage{array}                                          
    \usepackage{longtable}                                      
    \usepackage{calc}                                           
    \usepackage{multirow}                                       
    \usepackage{hhline}                                         
    \usepackage{ifthen}
    \usepackage{lscape}     
\usepackage{multicol}
\usepackage{chngcntr}

\DeclareMathOperator*{\Res}{Res}

\renewcommand\thesection{\arabic{section}}
\renewcommand\thesubsection{\thesection.\arabic{subsection}}
\renewcommand\thesubsubsection{\thesubsection.\arabic{subsubsection}}

\renewcommand\thesectiondis{\arabic{section}}
\renewcommand\thesubsectiondis{\thesectiondis.\arabic{subsection}}
\renewcommand\thesubsubsectiondis{\thesubsectiondis.\arabic{subsubsection}}

\hyphenation{op-tical net-works semi-conduc-tor}
\def\inputGnumericTable{}                                 %%

\lstset{
%language=C,
frame=single, 
breaklines=true,
columns=fullflexible
}

\begin{document}

\newtheorem{theorem}{Theorem}[section]
\newtheorem{problem}{Problem}
\newtheorem{proposition}{Proposition}[section]
\newtheorem{lemma}{Lemma}[section]
\newtheorem{corollary}[theorem]{Corollary}
\newtheorem{example}{Example}[section]
\newtheorem{definition}[problem]{Definition}

\newcommand{\BEQA}{\begin{eqnarray}}
\newcommand{\EEQA}{\end{eqnarray}}
\newcommand{\define}{\stackrel{\triangle}{=}}

\bibliographystyle{IEEEtran}

\providecommand{\mbf}{\mathbf}
\providecommand{\pr}[1]{\ensuremath{\Pr\left(#1\right)}}
\providecommand{\qfunc}[1]{\ensuremath{Q\left(#1\right)}}
\providecommand{\sbrak}[1]{\ensuremath{{}\left[#1\right]}}
\providecommand{\lsbrak}[1]{\ensuremath{{}\left[#1\right.}}
\providecommand{\rsbrak}[1]{\ensuremath{{}\left.#1\right]}}
\providecommand{\brak}[1]{\ensuremath{\left(#1\right)}}
\providecommand{\lbrak}[1]{\ensuremath{\left(#1\right.}}
\providecommand{\rbrak}[1]{\ensuremath{\left.#1\right)}}
\providecommand{\cbrak}[1]{\ensuremath{\left\{#1\right\}}}
\providecommand{\lcbrak}[1]{\ensuremath{\left\{#1\right.}}
\providecommand{\rcbrak}[1]{\ensuremath{\left.#1\right\}}}
\theoremstyle{remark}
\newtheorem{rem}{Remark}
\newcommand{\sgn}{\mathop{\mathrm{sgn}}}
\providecommand{\abs}[1]{\left\vert#1\right\vert}
\providecommand{\res}[1]{\Res\displaylimits_{#1}} 
\providecommand{\norm}[1]{\left\lVert#1\right\rVert}
%\providecommand{\norm}[1]{\lVert#1\rVert}
\providecommand{\mtx}[1]{\mathbf{#1}}
\providecommand{\mean}[1]{E\left[ #1 \right]}
\providecommand{\fourier}{\overset{\mathcal{F}}{ \rightleftharpoons}}
%\providecommand{\hilbert}{\overset{\mathcal{H}}{ \rightleftharpoons}}
\providecommand{\system}{\overset{\mathcal{H}}{ \longleftrightarrow}}
	%\newcommand{\solution}[2]{\textbf{Solution:}{#1}}
\newcommand{\solution}{\noindent \textbf{Solution: }}
\newcommand{\cosec}{\,\text{cosec}\,}
\providecommand{\dec}[2]{\ensuremath{\overset{#1}{\underset{#2}{\gtrless}}}}
\newcommand{\myvec}[1]{\ensuremath{\begin{pmatrix}#1\end{pmatrix}}}
\newcommand{\mydet}[1]{\ensuremath{\begin{vmatrix}#1\end{vmatrix}}}

\numberwithin{equation}{subsection}

\makeatletter
\@addtoreset{figure}{problem}
\makeatother

\let\StandardTheFigure\thefigure
\let\vec\mathbf

\renewcommand{\thefigure}{\theproblem}

\def\putbox#1#2#3{\makebox[0in][l]{\makebox[#1][l]{}\raisebox{\baselineskip}[0in][0in]{\raisebox{#2}[0in][0in]{#3}}}}
     \def\rightbox#1{\makebox[0in][r]{#1}}
     \def\centbox#1{\makebox[0in]{#1}}
     \def\topbox#1{\raisebox{-\baselineskip}[0in][0in]{#1}}
     \def\midbox#1{\raisebox{-0.5\baselineskip}[0in][0in]{#1}}
\vspace{3cm}

\title{Assignment 6}
\author{Surbhi Agarwal}

\maketitle

\newpage

%\tableofcontents

\bigskip

\renewcommand{\thefigure}{\theenumi}
\renewcommand{\thetable}{\theenumi}

\begin{abstract}
This document deals with QR decomposition and Singlular Value Decomposition
\end{abstract}

Download all python codes from 
%
\begin{lstlisting}
https://github.com/surbhi0912/EE5609/
\end{lstlisting}
%
and latex-tikz codes from 
%
\begin{lstlisting}
https://github.com/surbhi0912/EE5609/
\end{lstlisting}
%
\section{Problem}
1. Find the QR decomposition of $\vec{V} = \myvec{16 & 12\\12 & 9}$\\
2. Find the vertex of a parabola $(4x+3y+15)^2 = 5(3x-4y)$ using SVD and verify solution using least squares.
\section{Solution}
\subsection{QR decomposition of \vec{V}}
Let the column vectors of $\vec{V}$ be $\alpha$ and $\beta$:
\begin{align}
    \alpha=\myvec{16 \\ 12}\\
    \beta=\myvec{12 \\9}
\end{align}
We can express
\begin{align}
    \alpha = k_1\vec{u}_1 \label{eq:alpha}\\
    \beta = r_1\vec{u}_1+k_2\vec{u}_2\label{eq:beta}
    \intertext{where}
    k_1 = \norm\alpha = \sqrt{16^2+12^2} =20 \\
    \vec{u}_1 = \frac{\alpha}{k_1} = \frac{1}{20}\myvec{16 \\ 12} = \myvec{\frac{4}{5} \\ \frac{3}{5}}\\
    r_1 = \frac{\vec{u}_1^T\beta}{\norm{\vec{u}_1}^2} = 15\\
    \vec{u}_2 = \frac{\beta - r_1\vec{u}_1}{\norm{\beta - r_1\vec{u}_1}} = \frac{1}{0}\myvec{0\\0}
%%    k_2 = \vec{u}_2^T\beta = 0
%%\end{align}
%%From \eqref{eq:alpha} and \eqref{eq:beta},
%%\begin{align}
%%    \myvec{\alpha & \beta} = \myvec{\vec{u}_1 & \vec{u}_2}\myvec{k_1 & r_1\\0 & k_2}
%%    \intertext{where,}
%%    \vec{Q} = \myvec{\vec{u}_1 & \vec{u}_2}\\
%%    \vec{R} = \myvec{k_1 & r_1\\0 & k_2}
\end{align}
%%Here $\vec{R}$ is an upper triangular matrix and %%$\vec{Q}$ is an orthogonal matrix such that %%$\vec{Q}^T\vec{Q} = \vec{I}$. We get:
%%\begin{align}
%%    \vec{Q} = \myvec{\frac{4}{5} & 0\\ \frac{3}{5} & 0}\\
%%    \vec{R} = \myvec{20 & 15\\0 & 0}
%%\end{align}
%%Thus, $\vec{V} = \myvec{\frac{4}{5} & 0\\ \frac{3}{5} & 0}\myvec{20 & 15\\0 & 0}$
%%\subsection{Singular Value Decomposition for finding Vertex}
%%
\end{document}