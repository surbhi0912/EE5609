\documentclass[journal,12pt]{IEEEtran}

\usepackage{setspace}
\usepackage{gensymb}


\singlespacing

\usepackage[cmex10]{amsmath}
%\usepackage{amsthm}
%\interdisplaylinepenalty=2500
%\savesymbol{iint}
%\usepackage{txfonts}
%\restoresymbol{TXF}{iint}
%\usepackage{wasysym}
\usepackage{amsthm}

\usepackage{mathrsfs}
\usepackage{txfonts}
\usepackage{stfloats}
\usepackage{bm}
\usepackage{cite}
\usepackage{cases}
\usepackage{subfig}

\usepackage{longtable}
\usepackage{multirow}

\usepackage{enumitem}
\usepackage{mathtools}
\usepackage{steinmetz}
\usepackage{tikz}
\usepackage{circuitikz}
\usepackage{verbatim}
\usepackage{tfrupee}
\usepackage[breaklinks=true]{hyperref}

\usepackage{tkz-euclide} %loads TikZ and tkz-base

\usetikzlibrary{calc,math}
\usepackage{listings}
    \usepackage{color}                                          
    \usepackage{array}                                          
    \usepackage{longtable}                                      
    \usepackage{calc}                                           
    \usepackage{multirow}                                       
    \usepackage{hhline}                                         
    \usepackage{ifthen}
    \usepackage{lscape}     
\usepackage{multicol}
\usepackage{chngcntr}

\DeclareMathOperator*{\Res}{Res}

\renewcommand\thesection{\arabic{section}}
\renewcommand\thesubsection{\thesection.\arabic{subsection}}
\renewcommand\thesubsubsection{\thesubsection.\arabic{subsubsection}}

\renewcommand\thesectiondis{\arabic{section}}
\renewcommand\thesubsectiondis{\thesectiondis.\arabic{subsection}}
\renewcommand\thesubsubsectiondis{\thesubsectiondis.\arabic{subsubsection}}

\hyphenation{op-tical net-works semi-conduc-tor}
\def\inputGnumericTable{}                                 %%

\lstset{
%language=C,
frame=single, 
breaklines=true,
columns=fullflexible
}

\begin{document}

\newtheorem{theorem}{Theorem}[section]
\newtheorem{problem}{Problem}
\newtheorem{proposition}{Proposition}[section]
\newtheorem{lemma}{Lemma}[section]
\newtheorem{corollary}[theorem]{Corollary}
\newtheorem{example}{Example}[section]
\newtheorem{definition}[problem]{Definition}

\newcommand{\BEQA}{\begin{eqnarray}}
\newcommand{\EEQA}{\end{eqnarray}}
\newcommand{\define}{\stackrel{\triangle}{=}}

\bibliographystyle{IEEEtran}

\providecommand{\mbf}{\mathbf}
\providecommand{\pr}[1]{\ensuremath{\Pr\left(#1\right)}}
\providecommand{\qfunc}[1]{\ensuremath{Q\left(#1\right)}}
\providecommand{\sbrak}[1]{\ensuremath{{}\left[#1\right]}}
\providecommand{\lsbrak}[1]{\ensuremath{{}\left[#1\right.}}
\providecommand{\rsbrak}[1]{\ensuremath{{}\left.#1\right]}}
\providecommand{\brak}[1]{\ensuremath{\left(#1\right)}}
\providecommand{\lbrak}[1]{\ensuremath{\left(#1\right.}}
\providecommand{\rbrak}[1]{\ensuremath{\left.#1\right)}}
\providecommand{\cbrak}[1]{\ensuremath{\left\{#1\right\}}}
\providecommand{\lcbrak}[1]{\ensuremath{\left\{#1\right.}}
\providecommand{\rcbrak}[1]{\ensuremath{\left.#1\right\}}}
\theoremstyle{remark}
\newtheorem{rem}{Remark}
\newcommand{\sgn}{\mathop{\mathrm{sgn}}}
\providecommand{\abs}[1]{\left\vert#1\right\vert}
\providecommand{\res}[1]{\Res\displaylimits_{#1}} 
\providecommand{\norm}[1]{\left\lVert#1\right\rVert}
%\providecommand{\norm}[1]{\lVert#1\rVert}
\providecommand{\mtx}[1]{\mathbf{#1}}
\providecommand{\mean}[1]{E\left[ #1 \right]}
\providecommand{\fourier}{\overset{\mathcal{F}}{ \rightleftharpoons}}
%\providecommand{\hilbert}{\overset{\mathcal{H}}{ \rightleftharpoons}}
\providecommand{\system}{\overset{\mathcal{H}}{ \longleftrightarrow}}
	%\newcommand{\solution}[2]{\textbf{Solution:}{#1}}
\newcommand{\solution}{\noindent \textbf{Solution: }}
\newcommand{\cosec}{\,\text{cosec}\,}
\providecommand{\dec}[2]{\ensuremath{\overset{#1}{\underset{#2}{\gtrless}}}}
\newcommand{\myvec}[1]{\ensuremath{\begin{pmatrix}#1\end{pmatrix}}}
\newcommand{\mydet}[1]{\ensuremath{\begin{vmatrix}#1\end{vmatrix}}}

\numberwithin{equation}{subsection}

\makeatletter
\@addtoreset{figure}{problem}
\makeatother

\let\StandardTheFigure\thefigure
\let\vec\mathbf

\renewcommand{\thefigure}{\theproblem}

\def\putbox#1#2#3{\makebox[0in][l]{\makebox[#1][l]{}\raisebox{\baselineskip}[0in][0in]{\raisebox{#2}[0in][0in]{#3}}}}
     \def\rightbox#1{\makebox[0in][r]{#1}}
     \def\centbox#1{\makebox[0in]{#1}}
     \def\topbox#1{\raisebox{-\baselineskip}[0in][0in]{#1}}
     \def\midbox#1{\raisebox{-0.5\baselineskip}[0in][0in]{#1}}
\vspace{3cm}
\begin{center}
\huge Assignment 12\\
\large Surbhi Agarwal\\
\end{center}
\renewcommand{\thefigure}{\theenumi}
\renewcommand{\thetable}{\theenumi}

\begin{abstract}
This document illustrates concepts of dimensions of image of a linear transformation and columnspace.
\end{abstract}

\section{Problem}
Given a $4 \times 4$ matrix $\vec{A}$, let $T:\mathbb R^4 \rightarrow \mathbb R^4$ be the linear transformation defined by $\vec{T}\vec{v} = \vec{A}\vec{v}$, where we think of $\mathbb R^4$ as the set of real $4 \times 1$ matrices. For which choices of $\vec{A}$ given below, do Image$(\vec{T})$ and Image$(\vec{T}^2)$ have respective dimensions 2 and 1? ($*$ denotes a nonzero entry)
\begin{enumerate}
    \item $\vec{A} = \myvec{0 & 0 & * & * \\ 0 & 0 & * & * \\ 0 & 0 & 0 & * \\ 0 & 0 & 0 & 0}$
    \item $\vec{A} = \myvec{0 & 0 & * & 0 \\ 0 & 0 & * & 0 \\ 0 & 0 & 0 & * \\ 0 & 0 & 0 & *}$
    \item $\vec{A} = \myvec{0 & 0 & 0 & 0 \\ 0 & 0 & 0 & 0 \\ 0 & 0 & 0 & * \\ 0 & 0 & * & 0}$
    \item $\vec{A} = \myvec{0 & 0 & 0 & 0 \\ 0 & 0 & 0 & 0 \\ 0 & 0 & * & * \\ 0 & 0 & * & *}$
\end{enumerate}
\section{Solution}
We can say,
\begin{align}
    \vec{T}(\vec{v}) = \vec{A}\vec{v} = \text{Image}(\vec{T}) = C(\vec{A})\\
    \vec{T}^2(\vec{v}) = \vec{A}^2\vec{v} = \text{Image}(\vec{T}^2) = C(\vec{A}^2)
\end{align}
where $C(\vec{A})$ and $C(\vec{A}^2)$ denote the columnspace of $\vec{A}$ and $\vec{A}^2$ respectively. Therefore,
\begin{align}
    \text{dimension}(\text{Image}(\vec{T})) = \text{dimension}(C(\vec{A})) = \text{rank}(\vec{A})\\
    \text{dimension}(\text{Image}(\vec{T}^2)) = \text{dimension}(C(\vec{A}^2)) = \text{rank}(\vec{A}^2)
\end{align}
\renewcommand{\thetable}{1}
\begin{longtable}{|l|l|}
    \hline
        & \\
        1. $\vec{A} = \myvec{0 & 0 & * & * \\ 0 & 0 & * & * \\ 0 & 0 & 0 & * \\ 0 & 0 & 0 & 0}$ & The number of linearly independent columns in $\vec{A}$ is 2\\
    \hline
        & \\
        & hence, $dim(Image(\vec{T})) = dim(C(\vec{A})) = 2$\\
        & \\
        & $\vec{A}^2 = \myvec{0 & 0 & 0 & * \\ 0 & 0 & 0 & * \\ 0 & 0 & 0 & 0 \\ 0 & 0 & 0 & 0}$\\
        & \\
        & The number of linearly independent columns in $\vec{A}^2$ is 1\\
        & hence, $dim(Image(\vec{T}^2)) = dim(C(\vec{A}^2)) = 1$\\
        & \\
        & $\therefore$ This option is true.\\
    \hline
        & \\
        2. $\vec{A} = \myvec{0 & 0 & * & 0 \\ 0 & 0 & * & 0 \\ 0 & 0 & 0 & * \\ 0 & 0 & 0 & *}$ & The number of linearly independent columns in $\vec{A}$ is 2\\
        & hence, dim(Image($\vec{T}$)) = dim(C($\vec{A}$)) = 2\\
        & \\
        & $\vec{A}^2 = \myvec{0 & 0 & 0 & * \\ 0 & 0 & 0 & * \\ 0 & 0 & 0 & * \\ 0 & 0 & 0 & *}$\\
        & \\
        & The number of linearly independent columns in $\vec{A}^2$ is 1\\
        & hence, dim(Image($\vec{T}^2$)) = dim(C($\vec{A}^2$)) = 1\\
        & \\
        & $\therefore$ This option is true.\\
    \hline
        & \\
        3. $\vec{A} = \myvec{0 & 0 & 0 & 0 \\ 0 & 0 & 0 & 0 \\ 0 & 0 & 0 & * \\ 0 & 0 & * & 0}$ & The number of linearly independent columns in $\vec{A}$ is 2\\
        & hence, $dim(Image(\vec{T})) = dim(C(\vec{A})) = 2$\\
        & \\
        & $\vec{A}^2 = \myvec{0 & 0 & 0 & 0 \\ 0 & 0 & 0 & 0 \\ 0 & 0 & * & 0 \\ 0 & 0 & 0 & *}$\\
        & \\
        & The number of linearly independent columns in $\vec{A}^2$ is 2\\
        & hence, $dim(Image(\vec{T}^2)) = dim(C(\vec{A}^2)) = 2 \neq 1$\\
        & \\
        & $\therefore$ This option is false.\\
    \hline
        & \\
        4. $\vec{A} = \myvec{0 & 0 & 0 & 0 \\ 0 & 0 & 0 & 0 \\ 0 & 0 & * & * \\ 0 & 0 & * & *}$ & This option is false\\
        & Counter example:\\
        & For some non-zero $b,c \in \mathbb R$, let\\
        & $\vec{A} = \myvec{0 & 0 & 0 & 0 \\ 0 & 0 & 0 & 0 \\ 0 & 0 & b & b \\ 0 & 0 & c & c}$\\
        & \\
        & The number of linearly independent columns in $\vec{A}$ is 1\\
        & hence, $dim(Image(\vec{T})) = dim(C(\vec{A})) = 1 \neq 2$\\
        & \\
    \hline
    \caption{Verifying with the options}
    \label{tab:proof}
\end{longtable}
\end{document}