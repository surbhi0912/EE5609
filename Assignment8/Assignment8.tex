\documentclass[journal,12pt,twocolumn]{IEEEtran}

\usepackage{setspace}
\usepackage{gensymb}


\singlespacing

\usepackage[cmex10]{amsmath}
%\usepackage{amsthm}
%\interdisplaylinepenalty=2500
%\savesymbol{iint}
%\usepackage{txfonts}
%\restoresymbol{TXF}{iint}
%\usepackage{wasysym}
\usepackage{amsthm}

\usepackage{mathrsfs}
\usepackage{txfonts}
\usepackage{stfloats}
\usepackage{bm}
\usepackage{cite}
\usepackage{cases}
\usepackage{subfig}

\usepackage{longtable}
\usepackage{multirow}

\usepackage{enumitem}
\usepackage{mathtools}
\usepackage{steinmetz}
\usepackage{tikz}
\usepackage{circuitikz}
\usepackage{verbatim}
\usepackage{tfrupee}
\usepackage[breaklinks=true]{hyperref}

\usepackage{tkz-euclide} %loads TikZ and tkz-base

\usetikzlibrary{calc,math}
\usepackage{listings}
    \usepackage{color}                                          
    \usepackage{array}                                          
    \usepackage{longtable}                                      
    \usepackage{calc}                                           
    \usepackage{multirow}                                       
    \usepackage{hhline}                                         
    \usepackage{ifthen}
    \usepackage{lscape}     
\usepackage{multicol}
\usepackage{chngcntr}

\DeclareMathOperator*{\Res}{Res}

\renewcommand\thesection{\arabic{section}}
\renewcommand\thesubsection{\thesection.\arabic{subsection}}
\renewcommand\thesubsubsection{\thesubsection.\arabic{subsubsection}}

\renewcommand\thesectiondis{\arabic{section}}
\renewcommand\thesubsectiondis{\thesectiondis.\arabic{subsection}}
\renewcommand\thesubsubsectiondis{\thesubsectiondis.\arabic{subsubsection}}

\hyphenation{op-tical net-works semi-conduc-tor}
\def\inputGnumericTable{}                                 %%

\lstset{
%language=C,
frame=single, 
breaklines=true,
columns=fullflexible
}

\begin{document}

\newtheorem{theorem}{Theorem}[section]
\newtheorem{problem}{Problem}
\newtheorem{proposition}{Proposition}[section]
\newtheorem{lemma}{Lemma}[section]
\newtheorem{corollary}[theorem]{Corollary}
\newtheorem{example}{Example}[section]
\newtheorem{definition}[problem]{Definition}

\newcommand{\BEQA}{\begin{eqnarray}}
\newcommand{\EEQA}{\end{eqnarray}}
\newcommand{\define}{\stackrel{\triangle}{=}}

\bibliographystyle{IEEEtran}

\providecommand{\mbf}{\mathbf}
\providecommand{\pr}[1]{\ensuremath{\Pr\left(#1\right)}}
\providecommand{\qfunc}[1]{\ensuremath{Q\left(#1\right)}}
\providecommand{\sbrak}[1]{\ensuremath{{}\left[#1\right]}}
\providecommand{\lsbrak}[1]{\ensuremath{{}\left[#1\right.}}
\providecommand{\rsbrak}[1]{\ensuremath{{}\left.#1\right]}}
\providecommand{\brak}[1]{\ensuremath{\left(#1\right)}}
\providecommand{\lbrak}[1]{\ensuremath{\left(#1\right.}}
\providecommand{\rbrak}[1]{\ensuremath{\left.#1\right)}}
\providecommand{\cbrak}[1]{\ensuremath{\left\{#1\right\}}}
\providecommand{\lcbrak}[1]{\ensuremath{\left\{#1\right.}}
\providecommand{\rcbrak}[1]{\ensuremath{\left.#1\right\}}}
\theoremstyle{remark}
\newtheorem{rem}{Remark}
\newcommand{\sgn}{\mathop{\mathrm{sgn}}}
\providecommand{\abs}[1]{\left\vert#1\right\vert}
\providecommand{\res}[1]{\Res\displaylimits_{#1}} 
\providecommand{\norm}[1]{\left\lVert#1\right\rVert}
%\providecommand{\norm}[1]{\lVert#1\rVert}
\providecommand{\mtx}[1]{\mathbf{#1}}
\providecommand{\mean}[1]{E\left[ #1 \right]}
\providecommand{\fourier}{\overset{\mathcal{F}}{ \rightleftharpoons}}
%\providecommand{\hilbert}{\overset{\mathcal{H}}{ \rightleftharpoons}}
\providecommand{\system}{\overset{\mathcal{H}}{ \longleftrightarrow}}
	%\newcommand{\solution}[2]{\textbf{Solution:}{#1}}
\newcommand{\solution}{\noindent \textbf{Solution: }}
\newcommand{\cosec}{\,\text{cosec}\,}
\providecommand{\dec}[2]{\ensuremath{\overset{#1}{\underset{#2}{\gtrless}}}}
\newcommand{\myvec}[1]{\ensuremath{\begin{pmatrix}#1\end{pmatrix}}}
\newcommand{\mydet}[1]{\ensuremath{\begin{vmatrix}#1\end{vmatrix}}}

\numberwithin{equation}{subsection}

\makeatletter
\@addtoreset{figure}{problem}
\makeatother

\let\StandardTheFigure\thefigure
\let\vec\mathbf

\renewcommand{\thefigure}{\theproblem}

\def\putbox#1#2#3{\makebox[0in][l]{\makebox[#1][l]{}\raisebox{\baselineskip}[0in][0in]{\raisebox{#2}[0in][0in]{#3}}}}
     \def\rightbox#1{\makebox[0in][r]{#1}}
     \def\centbox#1{\makebox[0in]{#1}}
     \def\topbox#1{\raisebox{-\baselineskip}[0in][0in]{#1}}
     \def\midbox#1{\raisebox{-0.5\baselineskip}[0in][0in]{#1}}
\vspace{3cm}

\title{Assignment 8}
\author{Surbhi Agarwal}

\maketitle

\newpage

%\tableofcontents

\bigskip

\renewcommand{\thefigure}{\theenumi}
\renewcommand{\thetable}{\theenumi}

\begin{abstract}
This document deals with linear operators and basis of a finite dimensional vector space over a field.
\end{abstract}

\section{Problem}
Let $\mathbb V$ be finite dimensional vector space over the field $\mathbb F$, and let $\vec{S}$ and $\vec{T}$ be linear operators on $\mathbb V$. When do there exist ordered bases $\mathcal{B}$ and $\mathcal{B}'$ for $\mathbb V$ such that $[S]_\mathcal{B} = [T]_\mathcal{B}'$? Prove that such bases exist if and only if there is an invertible linear operator $\vec{U}$ on $\mathbb V$ such that $\vec{T} = \vec{U}\vec{S}\vec{U}^{-1}$
\section{Solution}
Assume $[S]_\mathcal{B} = [T]_\mathcal{B'}$, where $\vec{S}$ and $\vec{T}$ are linear operators on $\mathbb V$, and $\mathbb V$ has bases as follows
\begin{align}
    \mathcal{B} = \{\vec{x}_1,\vec{x}_2,\ldots,\vec{x}_n\}\nonumber\\
    \mathcal{B}' = \{\vec{x}_1',\vec{x}_2',\ldots,\vec{x}_n'\}\nonumber
\end{align}
Now, let $\vec{v} \in \mathbb V$. Then we can write $\vec{v}$ as a linear combination of the vectors of $\mathcal{B}$
\begin{align}
    \vec{v} = a_1\vec{x_1} + a_2\vec{x}_2 + \ldots + a_n\vec{x}_n
\end{align}
Let $\vec{U}$ be the operator which carries $\mathcal{B}$ to $\mathcal{B}'$. Then,
\begin{align}\label{eq:U}
    \vec{w} = \vec{U}(\vec{v}) = a_1\vec{x}_1' + a_2\vec{x}_2' + \ldots + a_n\vec{x}_n'
\end{align}
Given
\begin{align}
    [S]_\mathcal{B} = [T]_\mathcal{B}'\\
    \implies \vec{S}(\vec{v}) = c_1\vec{x_1} + c_2\vec{x}_2 + \ldots + c_n\vec{x}_n\quad\text{and}\label{eq:S}\\
    \vec{T}(\vec{w}) = c_1\vec{x}_1' + c_2\vec{x}_2' + \ldots + c_n\vec{x}_n'\label{eq:T}
\end{align}
Now using \eqref{eq:U},\eqref{eq:T} and \eqref{eq:S} in the following expresssion, we get
\begin{align}
    \vec{U}^{-1}\vec{T}\vec{U}(\vec{v}) & = \vec{U}^{-1}\vec{T}(\vec{w})\\&
    = \vec{U}^{-1}(c_1\vec{x}_1' + c_2\vec{x}_2' + \ldots + c_n\vec{x}_n') \\&
    = c_1\vec{x}_1 + c_2\vec{x}_2 + \ldots + c_n\vec{x}_n\\&
    = \vec{S}(\vec{v})\\
    \implies\vec{U}^{-1}\vec{T}\vec{U} = \vec{S}\\
    \implies \vec{T} = \vec{U}\vec{S}\vec{U}^{-1}
\end{align}
Hence, proved that there exist ordered bases $\mathcal{B}$ and $\mathcal{B}'$ for $\mathbb V$ such that if $[S]_\mathcal{B} = [T]_\mathcal{B}'$, then there is an invertible linear operator $\vec{U}$ on $\mathbb V$ such that $\vec{T} = \vec{U}\vec{S}\vec{U}^{-1}

Conversely, assume there exists an invertible operator $\vec{U}$ such that $\vec{T} = \vec{U}\vec{S}\vec{U}^{-1}$. Let
\begin{align}
    \mathcal{B} = \{\vec{x}_1,\vec{x}_2,\ldots,\vec{x}_n\}\nonumber\\
    \mathcal{B}' = \{\vec{U}(\vec{x}_1),\vec{U}(\vec{x}_2),\ldots,\vec{U}(\vec{x}_n)\}\nonumber
\end{align}
For $\vec{v} \in \mathbb V$,
\begin{align}
    \vec{v} = a_1\vec{x_1} + a_2\vec{x}_2 + \ldots + a_n\vec{x}_n
\end{align}
By definition,
\begin{align}
    \vec{T}(\vec{v}) & = \vec{T}(a_1\vec{x_1} + a_2\vec{x}_2 + \ldots + a_n\vec{x}_n)\\&
    = c_1\vec{x_1} + c_2\vec{x}_2 + \ldots + c_n\vec{x}_n \\&
    = \vec{U}\vec{S}\vec{U}^{-1}(\vec{v})
\end{align}
We know that $\vec{U}^{-1}(\vec{v}) = a_1\vec{x}_1 + a_2\vec{x}_2 + \ldots + a_n\vec{x}_n$. So $\vec{S}$ in basis $\mathcal{B}$ has the same entries as $\vec{T}$ in basis $\mathcal{B}'$.
Henced, proved that that there exist ordered bases $\mathcal{B}$ and $\mathcal{B}'$ for $\mathbb V$ such that if there is an invertible linear operator $\vec{U}$ on $\mathbb V$ such that $\vec{T} = \vec{U}\vec{S}\vec{U}^{-1}$, then $[S]_\mathcal{B} = [T]_\mathcal{B}'$
\end{document}